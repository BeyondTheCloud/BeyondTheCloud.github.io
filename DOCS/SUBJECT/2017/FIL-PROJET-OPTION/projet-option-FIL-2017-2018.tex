\documentclass[a4paper,11pt]{article}

% paquets perso
%\usepackage{clavierFonte}
%\usepackage{guillemets}
\usepackage[utf8]{inputenc}
\usepackage[T1]{fontenc}
\usepackage[french]{babel}
\newcommand{\fge}{\fg\ }


% paquet / note EMN
\usepackage{sujetProjet}

\usepackage{pdfpages}

\newcommand{\aegis}{\texttt{Aegis}}
\newcommand{\api}{\texttt{API}}
\newcommand{\cxf}{\texttt{Apache CXF}}
\newcommand{\ccxf}{\texttt{CXF}}
\newcommand{\corba}{\texttt{CORBA}}
\newcommand{\emn}{EMN}
\newcommand{\eclipse}{\texttt{Eclipse}}
\newcommand{\gsi}{GSI}
\newcommand{\discovery}{DisCoVery }
\newcommand{\http}{\texttt{HTTP}}
\newcommand{\java}{\texttt{Java}}
\newcommand{\jaxb}{\texttt{JAXB}}
\newcommand{\jaxws}{\texttt{JAX-WS}}
\newcommand{\jaxrs}{\texttt{JAX-RS}}
\newcommand{\jbi}{\texttt{JBI}}
\newcommand{\jee}{\texttt{JEE}}
\newcommand{\javaJEE}{\java{}/\jee{}}
\newcommand{\jms}{\texttt{JMS}}
\newcommand{\json}{\texttt{JSON}}
\newcommand{\restful}{\texttt{RESTful}}
\newcommand{\soap}{\texttt{SOAP}}
\newcommand{\uv}{UV}
\newcommand{\xml}{\texttt{XML}}




\initTitre{UV \og Sensibilisation à la recherche \fge \\ OpenStack et schémas d'accès SQL}
\initAbrevTitre{Projet Fog/Edge Computing}

%\initSousTitre{}

\initAuteurs{}

%\initDate{22}{8}{2006}

\initSousEntite{FIL}
\initCodeSousEntite{FIL}

%\initDestinataires{}

%\initDiffusion{}

%\initCodeNote{UV/}

%\initVersionNote{1.0}

%\initHistorique{}

\pasIndentation

\begin{document}

\begin{couverture}
  \afficherDate
  \afficherTitre

{\footnotesize
  \begin{sujet}

\afficherTuteur{%
\href{mailto:adrien.lebre@inria.fr}{Adrien Lebre}
}


\afficherProjet{%
 Fog/Edge Computing

 \begin{quote}
   OpenStack est une suite logiciel permettant d'opérer des
   plateformes de Cloud Computing de type IaaS. Bien que le passage à
   l'échelle (le terme de scalabilité est généralement utilisé) soit
   un critère décisif lors du choix d'un système pour opérer ce genre
   de plateforme, de nombreuses briques fondamentales constituants
   OpenStack ne satisfont pas ce critère, notamment lorsque vient
   s'associer au problème du passage à l'échelle celui de
   géo-distribution.  L'initiative Discovery pilotée par l'équipe
   ASCOLA éudie de manière approfondie le système OpenStack afin de
   déterminer les principales faiblesses et proposer des modifications
   permettant dy pallier.
 

  Le travail proposé dans ce projet s'inscrit dans cette action. il consiste à:
\begin{itemize}
\item Comprendre les traces JSON retournées par l'outil EnOS, outil utilisé pour l'évaluation des performanes du système OpenStack ;
\item Analyser ces traces de manière automatique afin de catégoriser les requêtes SQL réalisées par le système OpenStack ; 
\item De proposer comme résultat de cette analyse un ensemble de métriques (quantité de données manipulées par chaque catégorie, périodicité des requêtes, ...).
\end{itemize}

   A plus gros grain, ce projet s'insère dans l'initiative
\href{http://beyondtheclouds.github.io}{Discovery} pilotée par l'école. Le travail réalisé pourra donner lieu à une section dans un article de recherche. 
 \end{quote}
}

\afficherSiteWeb{http://beyondtheclouds.github.io}

\afficherIntitule{%
  Analyse des accès SQL du système OpenStack
  }

\end{sujet}
}
\end{couverture}


%\includepdf[pages=-, offset=2cm 0cm]{KeyCopyKey-projet.pdf}

\begin{note}


\paragraph*{Domaines}

\begin{itemize}
 \item Systèmes distribués
 \item Cloud Computing (IaaS)
 \item Python
\end{itemize}

\paragraph*{Compétences requises}

\begin{itemize}
 \item Fondamentaux Cloud Computing
 \item Programmation objets/composants
\end{itemize}

\paragraph*{Compétences à acquérir}

\begin{itemize}
\item Expertise sur l'architecture OpenStack
\item Sensibilation à l'évalution de performances de systèmes complexes
\item Contribution à un projet \og~open source~\fg
\item Programmation Python
\end{itemize}

\paragraph*{Contexte\\}
Afin de supporter la demande croissante d'informatique utilitaire (UC:
Utility Computing) tout en prenant en compte les aspects énergétiques
et financiers, la tendance actuelle consiste à construire des centres
de données (ou centrales numériques) de plus en plus grands dans un
nombre limité de lieux stratégiques.  Cette approche permet sans aucun
doute de satisfaire la demande tout en conservant une approche
centralisée de la gestion de ces ressources mais elle reste loin de
pouvoir fournir des infrastructures de calcul utilitaire efficaces et
durables. L'initiative de recherche \discovery propose une approche
radicalement différente permettant de prendre en compte la localité
des demandes dès le départ. Pour ce faire, elle propose de tirer parti
de tous les équipements disponibles sur l'Internet afin de fournir des
infrastructures de calcul utilitaire qui permettront de part leur
distribution de prendre en compte plus efficacement la dispersion
géographique des utilisateurs et leur demande toujours croissante
[1]. Un des aspects critique pour l'émergence de telles plates-formes
de calcul utilitaire (peu à peu connues sous les termes Fog et Edge
Computing) est la disponibilité de mécanismes de gestion appropriés
exploitant les dernières contributions en terme d'algorithmiques
distribués.  Des premiers travaux menés à l'école des Mines ont permis
de valider plusieurs mécanismes via la mise en \oe uvre de prototypes
experimentaux. Afin de passer du mode recherche à un mode plus
industriel, nous avons commencé à regarder comment intégrer nos
propositions au système OpenStack.



\paragraph*{Problème: Les faiblesses dans la conception d'OpenStack\\}

OpenStack est une suite logiciel permettant d'opérer des plateformes
de Cloud Computing de type IaaS. Elle est aujourd'hui la solution
\emph{Open Source} retenue et supportée par un grand nombre d'acteurs aussi
bien académiques et industriels (850 institutions sont impliquées
dans son développement, avec des acteurs de premier rangs tels que IBM,
RedHat ou encore plus récemment Google).  Bien qu'une des règles
fondamentales pour le développement du système a pour objectif
d'adresser la problématique du passage à l'échelle de chaque
composant, le système comprend plusieurs briques logicielles qui ne
repondent pas à cette règle. A titre d'exemple, les travaux conduits à l'école ont identifié la base de
données MySQL au sein du composant Nova. Bien qu'il soit possible comme nous l'avons montré de développer des prototypes
au dessus des systèmes NoSQL [1] (ou plus récemment NewSQL), il est nécessaire de comprendre l'implication du modèle SQL au sein du code propre au système OpenStack.


\paragraph*{Objectif\\}
L'objectif du travail proposé dans ce projet vise à étendre la phase
d'analyse post-mortem de l'outil
EnOS\footnote{https://github.com/BeyondTheClouds/enos} [2] développé dans
le cadre de l'initiative Discovery. Cet outil permet de
deployer une infrastructure OpenStack au dessus d'une large varièté de
plateformes experimentales grâce à la technologie Docker et Ansible.
Une fois le système déployé (phase 1), il est possible d'éxecuter un ensemble de tests de performance (phase 2) et d'analyser les résultats de manière post-mortem (phase 3).
L'objectif du projet est d'enrichir la phase d'analyse de manière à donner une vue synthétique des informations relatives aux accès SQL réalisés.

Le projet sera articulé autour de trois volets (le dernier pouvant être vu comme un bonus):
\begin{itemize}
\item La lecture et la compréhension d'articles scientifiques et de documentation technique (référencés ci-après) ; 
\item L'analyse des requêtes SQL contenues dans la trace JSON delivrée par EnOS (et plus précisement le module OSProfiler) ; 
\item La mise en place d'une vue synthétique (de type dashboard) permettant de parcourir les informations collectées.  
\end{itemize}

\paragraph*{Méthodologie du projet\\}
Ce projet sera réalisé suivant une méthode agile. Il devra impliquer une
contribution au projet \emph{Open Source} EnOS.


\paragraph*{Bibliographie:\\}
%
\texttt{[1]} Revising OpenStack to Operate Fog/Edge Computing infrastructure, Lebre et al., In Proceedings of the IEEE Internation conference IC2E, Vancouver, April 2017 - \url{https://hal.inria.fr/hal-01273427/}.

\texttt{[2]}ENOS: a Holistic Framework for Conducting Scientific Evaluations of OpenStack, Cherrueau et al., in Proceedings of the IEEE/ACM Interntional Conference CCGRID, May 2017 - \url{https://hal.inria.fr/hal-01415522v2/}.

\end{note}



\end{document}

\documentclass[a4paper,11pt]{article}

% paquets perso
%\usepackage{clavierFonte}
%\usepackage{guillemets}
\usepackage[utf8]{inputenc}
\usepackage[T1]{fontenc}
\usepackage[french]{babel}
\newcommand{\fge}{\fg\ }


% paquet / note EMN
\usepackage{sujetProjet}

\usepackage{pdfpages}

\newcommand{\aegis}{\texttt{Aegis}}
\newcommand{\api}{\texttt{API}}
\newcommand{\cxf}{\texttt{Apache CXF}}
\newcommand{\ccxf}{\texttt{CXF}}
\newcommand{\corba}{\texttt{CORBA}}
\newcommand{\emn}{EMN}
\newcommand{\eclipse}{\texttt{Eclipse}}
\newcommand{\gsi}{GSI}
\newcommand{\discovery}{DisCoVery }
\newcommand{\http}{\texttt{HTTP}}
\newcommand{\java}{\texttt{Java}}
\newcommand{\jaxb}{\texttt{JAXB}}
\newcommand{\jaxws}{\texttt{JAX-WS}}
\newcommand{\jaxrs}{\texttt{JAX-RS}}
\newcommand{\jbi}{\texttt{JBI}}
\newcommand{\jee}{\texttt{JEE}}
\newcommand{\javaJEE}{\java{}/\jee{}}
\newcommand{\jms}{\texttt{JMS}}
\newcommand{\json}{\texttt{JSON}}
\newcommand{\restful}{\texttt{RESTful}}
\newcommand{\soap}{\texttt{SOAP}}
\newcommand{\uv}{UV}
\newcommand{\xml}{\texttt{XML}}




\initTitre{UV \og Projet \fge -- Sujet DisOStack  \\ Distribution de
  l'architecture OpenStack}
\initAbrevTitre{Projet DIsOStack}

%\initSousTitre{}

\initAuteurs{}

%\initDate{22}{8}{2006}

\initSousEntite{Option GSI}
\initCodeSousEntite{GSI}

%\initDestinataires{}

%\initDiffusion{}

%\initCodeNote{UV/}

%\initVersionNote{1.0}

%\initHistorique{}

\pasIndentation

\begin{document}

\begin{couverture}
  \afficherDate
  \afficherTitre
\begin{sujet}

\afficherTuteur{%
\href{mailto:adrien.lebre@inria.fr}{Adrien Lebre}
}


\afficherProjet{%
 DisOStack

 \begin{quote}
  OpenStack est une suite logiciel permettant d'opérer des plateformes
  de Cloud Computing de type IaaS.
  Bien le passage à l'échelle (i.e. scalabilité) soit un critère décisif lors du 
  choix d'un système pour opérer ce genre de plateforme, et bien
  que l'accent soit mis sur la scalabilité au cours du développment, 
  de nombreuses briques fondamentales constituants les IaaS contemporains ne 
  satisfont pas ce critère, impactant negativement la 
  scalabilité globale.

  Le travail proposé comporte deux objectifs:
\begin{itemize}
\item L'identification de l'ensemble des briques élémentaires ne
  validant la propriété de scalabilité
\item la proposition d'une solution et sa mise en \oe uvre au travers
  d'un prototype experimental (selon l'avancée des travaux réalisées,
  les étudiants pourront mener des campagnes d'experiences sur la
  plateforme \href{Grid'5000}{Grid'5000}).
\end{itemize}

A plus gros grain, ce projet s'insère dans l'initiative
\href{http://beyondtheclouds.github.io}{Discovery} piloté par l'école.
 \end{quote}
}

\afficherSiteWeb{http://beyondtheclouds.github.io}

\afficherIntitule{%
 Analyse du système OpenStack en vue de sa distribution à large
 échelle.
}

\end{sujet}

\end{couverture}


%\includepdf[pages=-, offset=2cm 0cm]{KeyCopyKey-projet.pdf}

\begin{note}


\paragraph*{Domaines}

\begin{itemize}
 \item Intégration logicielle
 \item Cloud Computing
 \item Modele de données (SQL, NoSQL, Clé/Valeur)
\end{itemize}

\paragraph*{Compétences requises}

\begin{itemize}
 \item Fondamentaux Cloud Computing
 \item Maîtrise des modèles de données relationnels.
 \item Programmation objets/composants
\end{itemize}

\paragraph*{Compétences à acquérir}

\begin{itemize}
 \item Expertise sur l'architecture OpenStack
 \item Langage python
 \item Prise en main de l'EDI \texttt{IntelliJ}
 \item Développement de schémas de données NoSQL
 \item Contribution à un projet \og~open source~\fg
\end{itemize}

\paragraph*{Contexte\\}

 Afin de supporter la demande croissante d'informatique utilitaire (UC: Utility Computing) tout en prenant en compte les aspects énergétique et financier, la tendance actuelle consiste à construire des centres de données (ou centrales numériques) de plus en plus grands dans un nombre limité de lieux stratégiques. Cette approche permet sans aucun doute de satisfaire la demande tout en conservant une approche centralisée de la gestion de ces ressources mais elle reste loin de pouvoir fournir des infrastructures de calcul utilitaire efficaces et durables.
   L'initiative de recherche \discovery propose une approche radicalement différente permettant de prendre en compte la localité des demandes dès le départ.
Pour ce faire, elle propose de tirer parti de tous les équipements
disponibles sur l'Internet afin de fournir des infrastructures de calcul
utilitaire  qui permettront de part leur distribution de prendre en compte plus
efficacement la dispersion géographique des utilisateurs et leur demande
toujours croissante [1]. Un des aspects critique pour l'émergence de telles
plates-formes de calcul utilitaire ''local'' (LUC: Locality aware Utility Computing) est la disponibilité de
mécanismes de gestion appropriés exploitant les dernières
contributions en terme d'algorithmiques distribués.
Des premiers travaux menés à l'école des Mines ont permis de valider
plusieurs mécanismes via la mise en \oe uvre de prototypes
experimentaux.
Afin de passer du mode recherche à un mode plus industriel, nous
avons commencé à regarder comment intégrer nos propositions au système
OpenStack.

\paragraph*{Problème: Les faiblesses dans la conception d'OpenStack\\}

  OpenStack est une suite logiciel permettant d'opérer des plateformes
  de Cloud Computing de type IaaS. Elle est aujourd'hui la solution
  OpenSource retenue et supportée par un grand nombre d'acteurs
  aussi bien académiques et qu'industriels (850 institutions sont
  impliquées dans son développement, avec des acteurs de premier rangs
  tels que HP ou encore RedHat).
  Bien qu'une des règles fondamentales pour le développement du
  système a pour objectif d'adresser la problématique du passage à
  l'échelle de chaque composant (le terme scalabilité sera utilisé par
  la suite), le système comprend plusieurs briques
  logicielles qui ne repondent pas à
  cette règle. A titre d'exemple, Jonathan Pastor, actuellement en
  thèse de doctorat à l'école, a identifié la base de données MySQL au
  sein du composant Nova et a débuté des travaux visant à basculer sur
  un modèle NoSQL passant à l'échelle de manière native.


\paragraph*{Objectif\\}

  L'objectif du travail proposé dans ce projet d'option s'insère dans
  la continuité de ces travaux. Il comporte deux
  volets:
\begin{itemize}
\item L'identification de l'ensemble des briques élémentaires ne
  validant le critère de scalabilité.
\item la proposition d'une solution et sa mise en \oe uvre au travers
  d'un prototype experimental (selon l'avancée des travaux réalisées,
  les étudiants pourront mener des campagnes d'experiences sur la
  plateforme \href{Grid'5000}{Grid'5000}).
\end{itemize}

\paragraph*{Livrables\\}

\begin{itemize}
\item L0 : rapport sur l'architecture d'OpenStack (ces principaux
  composants et leur rôle).
\item L1 : identification des briques logicielles ne respectant par la
  règle du passage à l'échelle.
\item L2 : Proposition d'une solution pour un des composants choisis
  avec le tuteur.
\item L3 : implémentation de la solution et évualuation sur la
  plateforme Grid'5000.
\item L4 : synthèse des travaux réalisés.
\end{itemize}

\paragraph*{Méthodologie du projet\\}
Ce projet sera réalisé suivant une méthode agile. Il devra impliquer une
contribution au projet \og~open source~\fg\ Discovery.


\paragraph*{Bibliographie:\\}
%
\texttt{[1]} Adrien Lebre, Jonathan Pastor, Marin Bertier, Frédéric Desprez, Jonathan Rouzaud-Cornabas, Cédric Tedeschi, Paolo Anedda, Gianluigi Zanetti, Ramon Nou, Toni Cortes, Etienne Rivière, and Thomas Ropars. Beyond The Cloud, How Should Next Generation Utility Computing Infrastructures Be Designed? Research Report RR-8348, Inria, July 2013.\\

\end{note}



\end{document}

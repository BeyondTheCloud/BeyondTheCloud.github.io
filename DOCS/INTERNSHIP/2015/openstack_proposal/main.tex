\documentclass{letter}
\usepackage{style-ltr}
\usepackage{graphicx}
\usepackage{hyperref}


\signature{Adrien Lebre}
\begin{document}
\name{Dr.\ Adrien lebre}
\position{Researcher\\(currently on leave from Mines Nantes)}
\address{B216 - Ecole des Mines de Nantes\\
4, rue Alfred Kastler, BP 20722\\
44307 Nantes Cedex 3,\\
France}
\email{adrien.lebre@inria.fr}


\begin{letter}
{\Large \textbf{Master Internship}}


\date{01 March 2015}

\opening{}

{\textbf{Title}: Participation to the development of a fully distributed
 IaaS manager based on OpenStack.\\
\textbf{Duration}: 4 to 6 months\\
\textbf{Location}: Ecole des Mines de Nantes (France)\\
\textbf{Supervisors}: \begin{itemize}
\item Adrien Lebre (adrien.lebre@inria.fr)
\item Jonathan Pastor (jonathan.pastor@inria.fr)
\end{itemize}
}

{\Large \textbf{I- Context of the Internship}}

In the current ecosystem of the Internet, service-providers are offering services
to their users from all over the world. These services require
a large amount of computing resources, and each service-provider has its own computing
infrastructure leveraging tens of thousand servers concentrated in large 
data-centers (DCs). 

However, concentrating the production of computing resources leads to several 
issues:

\begin{itemize}
\item  Writing scalable software to manage such large scale infrastructures is
difficult.
\item Large DCs require dedicated electrical and cooling facilities.
\item Providing services to far-away users is source of network overhead.
\item World-wide infrastructures leads to jurisdictional conflicts.
\end{itemize}

The DISCOVERY initiative proposes to change this model of \textbf{"few large 
data-centers"} to \textbf{"many small data-centers, geographically spread, deployed upon 
the network backbones"} in order to benefit from existing network centers, starting from the 
core nodes of the backbone to the different network access points in charge of 
interconnecting public and private institutions. 

By such a mean, network and cloud providers would be able to mutualize resources
that are mandatory to operate network/data centers while delivering widely 
distributed cloud computing platforms being able to better match the 
geographical dispersal of users.

More information can be found at 
\href{http://beyondtheclouds.github.io}{http://beyondtheclouds.github.io}.


{\Large \textbf{II- Description of the Internship}}

In a first time the intern will study the functionning of OpenStack: he will
identify the interactions between its constituting services, in order to 
understand how they collaborate. In parallel he will take cognizance of the
approach proposed by Jonathan Pastor, a Ph.D student of the Ascola team.

In a second time, the intern will focus his work on distributing the Glance 
image service, which is in charge of managing the images used by the virtual 
machines that are provisioned by an OpenStack deployment. Leveraging the work 
accomplished during the first stage, the intern will migrate the database 
backend from a MySQL solution to a non relational database such as RIAK.

Researchers of the Ascola team are currently working on building a fully 
distributed IaaS manager based on the OpenStack project. A prototype is actually
tested on the Grid'5000 testbed. In the last stage, the intern will 
integrate its work in this prototype, in order to validate the proposed 
approach. To facilitate this part, the intern will work closely with Jonathan 
Pastor, in particular for the validation on Grid'5000.

{\Large \textbf{III- Required skills}}

\begin{itemize}
\item Curiosity and inquiring spirit.
\item Good algorithmic skills.
\item Knowledge in Python and Scala is a plus.
\item Knowledge in web programming is a plus.
\end{itemize}

\end{letter}
\end{document}

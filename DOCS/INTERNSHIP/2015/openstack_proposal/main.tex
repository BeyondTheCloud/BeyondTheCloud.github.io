\documentclass{letter}
\usepackage{cornell-ltr}
\usepackage{graphicx}


\signature{Adrien Lebre}
\begin{document}
\name{Dr.\ Adrien lebre}
\position{Researcher\\(currently on leave from Mines Nantes)}
\address{B216 - Ecole des Mines de Nantes\\
4, rue Alfred Kastler, BP 20722\\
44307 Nantes Cedex 3,\\
France}
\email{adrien.lebre@inria.fr}


\begin{letter}
{\Large \textbf{Master Internship}}


\date{01 March 2015}

\opening{}

{\textbf{Title}: Participation to the development of a fully distributed
 IaaS manager based on OpenStack.\\
\textbf{Duration}: 4 to 6 months\\
\textbf{Location}: Ecole des Mines de Nantes (France)\\
\textbf{Supervisors}: \begin{itemize}
\item Adrien Lebre (adrien.lebre@inria.fr)
\item Jonathan Pastor (jonathan.pastor@inria.fr)
\end{itemize}
}

{\Large \textbf{I- Context of the Internship}}

In the current ecosystem of the Internet, service-providers are offering services
to their users from all over the world. These services require
a large amount of computing resources, and each service-provider has its own computing
infrastructure leveraging tens of thousand servers concentrated in large 
data-centers (DCs). 

However, concentrating the production of computing resources leads to several 
issues:

\begin{itemize}
\item  Writing scalable software to manage large scale infrastructures is
difficult.
\item Large DCs requires dedicated electrical and cooling facilities.
\item Providing services to far-away users causes a network overhead.
\item World-wide infrastructures leads to jurisdictional conflicts.
\end{itemize}

The DISCOVERY initiative proposes to change this model of \textbf{"few large 
data-centers"} to \textbf{"many small data-centers, geographically spread, deployed upon 
the network backbones"} in order to benefit from existing network centers, starting from the 
core nodes of the backbone to the different network access points in charge of 
interconnecting public and private institutions. 

By such a mean, network and cloud providers would be able to mutualize resources
that are mandatory to operate network/data centers while delivering widely 
distributed cloud computing platforms being able to better match the 
geographical dispersal of users.

{\Large \textbf{II- Description of the Internship}}

In a first time the intern will study the functionning of OpenStack: he will
identify the interactions between the composing services of OpenStack, in
order to indentify which mechanisms can be distributed.

In a second time, the intern will work directly on a prototype of an OpenStack 
based IaaS manager. The intern will demonstrate the results of its previous work
by applying it on the prototype. During this stage, the intern will work in team
with some members of the Discovery initiative.

\end{letter}
\end{document}

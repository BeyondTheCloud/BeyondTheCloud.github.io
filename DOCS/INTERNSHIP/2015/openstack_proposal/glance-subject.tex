\documentclass{letter}
\usepackage{style-ltr}
\usepackage{graphicx}
\usepackage{hyperref}


\signature{Adrien Lebre}
\begin{document}
\name{Dr.\ Adrien lebre}
\position{Inria Researcher}
%\\(on leave from an Ass. Prof  position at Mines Nantes)}
\address{B216 - Ecole des Mines de Nantes\\
4, rue Alfred Kastler, BP 20722\\
44307 Nantes Cedex 3,\\
France}
%\email{adrien.lebre@inria.fr}


\begin{letter}
{\Large \textbf{Master Internship Position}}


\date{}

\opening{}

{\textbf{Title}: Participation to the development of a fully distributed
 IaaS manager based on OpenStack.\\
\textbf{Application deadline: 15 April 2015}
\\
\textbf{Duration}: 4 to 6 months\\
\textbf{Location}: Ecole des Mines de Nantes (France)\\
\textbf{Allowance}: 500 Euros per month\\
\textbf{Supervisors}: \begin{itemize}
\item Adrien Lebre (adrien.lebre@inria.fr)
\item Jonathan Pastor (jonathan.pastor@inria.fr)
\end{itemize}
}

{\Large \textbf{Context of the Internship}}

In the current ecosystem of the Internet, service-providers are offering services
to their users from all over the world. These services require
a large amount of computing resources, and each service-provider has its own computing
infrastructure leveraging tens of thousand servers concentrated in large
data-centers (\textit{a.k.a., } mega-DCs).

However, concentrating the production of computing resources in few
mega-DCs  leads to several
issues:

\begin{itemize}
\item  Designing and implementing dedicated software to manage such large scale infrastructures is
difficult.
\item mega-DCs require dedicated electrical and cooling facilities.
\item Providing services to users far-away  is source of network overhead.
\item World-wide infrastructures leads to jurisdictional conflicts.
\end{itemize}

The DISCOVERY initiative proposes an alternative to the mega-DC model.
Instead of building DCs of ever-increasing size, the initiative
proposes to build many small data-centers, geographically spread,
deployed upon the network backbones, taking the advantages of existing
network centers, starting from the core nodes of the backbone to the
different network access points in charge of interconnecting public
and private institutions.

By such a mean, network and cloud providers would be able to mutualize resources
that are mandatory to operate network/data centers while delivering widely
distributed cloud computing platforms being able to better match the
geographical dispersal of users.

More information can be found on the
\href{http://beyondtheclouds.github.io}{DISCOVERY website}.


{\Large \textbf{Description of the Internship}}

The internship will be integrated in the ASCOLA research group located
at the Ecole des Mines de Nantes (further information on the
\href{http://www.emn.fr/z-info/ascola/doku.php}{ASCOLA website}).
The objectives of this internship are structured around the on-going
development ASCOLA is doing for the aforementionned initiative.
Concretely, ASCOLA is currently revisiting the design of the OpenStack
solution in order to deliver a fully distributed IaaS manager. A
prototype is currently tested on the Grid'5000 testbed.

After studying the OpenStack system (identifying the interactions between its constituting services in order to
understand how they collaborate), the student will focus on distributing the Glance
image service, which is in charge of managing the images used by the virtual
machines that are provisioned by an OpenStack deployment.
At coarse grained, the work will consist in migrating the database
backend from a MySQL solution to a non relational database such as RIAK.

 In the last stage, the intern will integrate his/her work in the
 prototype developped by ASCOLA, in order to validate the proposed
approach. To facilitate this part, the intern will work closely with Jonathan
Pastor, a 3rd year PhD applicant at Mines Nantes, in particular for the validation on Grid'5000.

{\Large \textbf{Required skills}}

\begin{itemize}
\item Curiosity and inquiring spirit.
\item Good algorithmic skills.
\item Knowledge in Python and Scala is a plus.
%\item Knowledge in web programming is a plus.
\end{itemize}

\end{letter}
\end{document}

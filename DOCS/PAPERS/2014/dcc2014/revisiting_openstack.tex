\section{Revisiting existing cloud mechanisms}
\label{sec:integration}

\subsection{Openstack: a toolkit for building cloud}

\begin{itemize}
	
	\item Opensource project considered as a standard in cloud infrastrcutre.

	\item Composed of several "shared nothing architecture" services (nova, 
	swift, Quantum, Glance, ...).

	\item Uses an AMQP (Advanced Message Queuing Protocol) for inter-components 
	communication. => additional components can easily plug on an existing
	infrastructure.

	\item Drawack: Openstack does not perform dynamic scheduling of VMs.

\end{itemize}


\subsection{Revisiting components that are suitable for a LUC-OS (Related work)}

\begin{itemize}

	\item As we do not want to redevelop every services "from scratch", we will
	maximize the reuse of existing	mechanisms.

	\item In section \ref{sec:architecture} we defined a list of services provided 
	by the LUC-OS. Most of them will partially/completly leverage existing 
	mechanisms:

		\begin{description}

			\item [Compute manager] : 
			\begin{itemize}
				\item Nova is the service that is responsible for controlling 
				every services of the infrastucture. 

				\item It leverages nova-scheduler, which performs static 
				scheduling: nova-scheduler.

				\item We plan to use a dynamic scheduler (DVMS) instead.
			\end{itemize}

			\item [Administrative manager] : 
			\begin{itemize}
				\item KeyStone is the service used for managing identity and
				authentication.

				\item Horizon is the service that serves a web interface, 
				allowing users to interact with OpenStack.

				\item As we will use a distributed hash table (DHT) to save
				persistent states, we plan reuse Keystone and Horizon, with
				minor modification to leverage our DHT. 
			\end{itemize}

			\item [Storage manager] : 
			\begin{itemize}
				\item Swift is a distributed key/value service with no SPOF.
				
				\item Glance is a service for storing image that leverages 
				Swift. It is possible to directly store images in Swift.

				\item To limit network impact ("Boot problem"), solution based 
				on bittorrent have been developped (VMTorrent, 
				glance-bittorrent, ...).

				\item For a first prototype we propose to leverage Glance and
				Swift. If this solution become no more suitable for a massive
				infrastucture, we propose to integrate VMTorrent in Glance.

			\end{itemize}

			\item [Network manager] :
			\begin{itemize}
				
				\item Neutron is the service used to create VLAN.

				\item It supports plugins: it is possible to use Open vSwitch.
				This is what we propose for a first version of the system.

				\item For next versions, Mininet \ref{lanz:2010} or Vin can be
				considered as good candidates.
				
			\end{itemize}

		\end{description}

\end{itemize}


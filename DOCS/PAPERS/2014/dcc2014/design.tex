\section{Designing a massively distributed Cloud}
\label{sec:design}

\subsection{Massively distributed cloud}

\begin{itemize}

	\item A massively distributed cloud targets management of thousand of hosts 
	around a wide territory.

	\item This scale order is currently reached by file sharing systems like 
	bittorrent.

	\item At this scale, failure becomes the norm.

	\item Recent works propose to leverage on peer to peer overlay.

	\item Some peer to peer overlays can take advantage of locality. It enables
	to build systems that can take into account network bandwidth and latency.

	\item We propose to leverage on locality based peer to peer mechanisms to 
	reach an high scalability IaaS.

\end{itemize}

\subsection{Toolkit for IaaS}

\begin{itemize}

	\item A toolkit is a building block that can be used for the construction of
	systems (generic definition of a software toolkit).

	\item The objective of a toolkit is to provide "state of the art" solutions
	to known problems. It enables the focus on "Top level" works.

	\item It provides a set of components, which once assembled constitute an 
	operational system.

	\item Recent studies of "state of art IaaS systems" (Openstack, Cloudstack,
	OpenNebula, ...) showed that they were constructed over same concepts. It 
	enables the design of IaaS toolkit.

	\item The massively distributed IaaS toolkit will provides "state of the 
	arts" mechanism to solve both scalability and locality points.

	\item The toolkit will have to integrate well on existing systems: we 
	propose to leverage Openstack project.

\end{itemize}

\section{Designing a massively distributed Cloud}
\label{sec:design}


\subsection{Toolkit for IaaS}

% \begin{itemize}

% 	\item A toolkit is a building block that can be used for the construction of
% 	systems (generic definition of a software toolkit).

% 	\item The objective of a toolkit is to provide "state of the art" solutions
% 	to known problems. It enables the focus on "Top level" works.

% 	\item It provides a set of components, which once assembled constitute an 
% 	operational system.

% 	\item Recent studies of "state of art IaaS systems" (OpenStack, Cloudstack,
% 	OpenNebula, ...) showed that they were constructed over same concepts. It 
% 	enables the design of IaaS toolkit.

% 	\item The massively distributed IaaS toolkit will provides "state of the 
% 	arts" mechanism to solve both scalability and locality points.

% 	\item The toolkit will have to integrate well on existing systems: we 
% 	propose to leverage OpenStack project.

% \end{itemize}

A software toolkit is a set of software building blocks that includes state of 
the art mechanisms for known problems. A toolkit comes with an API (Application
Programming Interface) which is the specifications that ones must follow to
correctly use provided mechanisms. The goal of a toolkit is to enable developer
to focus on the creation of higher level mechanisms, thus speeding up the 
development time.

An IaaS toolkit should be delivered with a set of default high level mechanisms 
whose assembly results in an basic operational IaaS system. If the case where 
one of the default constituting mechanisms would not be sufficient, it should be
redeveloped by leveraging the toolkit's low level mechanisms.

Recent studies of state of the art IaaS manager \cite{peng:2009} show that they
were constructed over the same concepts. Furthermore a reference architecture 
for IaaS manager has been described in \cite{moreno2012iaas} enabling the design
of an IaaS toolkit. Besides the reference architecture, an IaaS toolkit should
provides some mechanisms that enables to solve the scalability problem.

To maximize the chance of being reuse by a large community, an IaaS toolkit 
should enable an easy integration with one or several existing IaaS cloud
managers. In our case we have chosen to leverage the OpenStack project: as a 
result the mechanisms developped with the toolkit will integrate well with 
existing clouds that are based on OpenStack.



\subsection{Massively distributed cloud}

\begin{itemize}

	\item A massively distributed cloud targets management of thousand of hosts 
	around a wide territory.

	\item This scale order is currently reached by file sharing systems like 
	bittorrent.

	\item At this scale, failure becomes the norm.

	\item Recent works propose to leverage on peer to peer overlay.

	\item Some peer to peer overlays can take advantage of locality. It enables
	to build systems that can take into account network bandwidth and latency.

	\item We propose to leverage on locality based peer to peer mechanisms to 
	reach an high scalability IaaS.

\end{itemize}


%\subsection{Efficient Virtual Machine Management}

Many virtual infrastructure managers have been proposed to deal with specific concerns.
In this section, we will focus on some of their limitations, especially regarding
locality, scalability, and fault-tolerance.

The most common managers are the centralized ones, like
Entropy~\cite{hermenier:cp11,hermenier:2013}, since they are easy to deploy.
%
They are generally designed to work on a cluster.
%
In this context, they do not take account of the network topology, and they
cannot manage VMs efficiently in a multi-site/multi-cluster deployment.
%
Moreover, they are prone to fault-tolerance, scalability, and reactivity issues; to avoid
these limitations, one possibility is to rely on more decentralized approaches, like
hierarchical or distributed ones.

Hierarchical managers, like Snooze~\cite{feller:ccgrid12}, may be more suited to handle
locality.
%
For instance, it is possible to setup (i)~one manager per cluster, and (ii)~one
(fault-tolerant) super manager that monitors cluster managers and chooses on which cluster
a new VM should start.
%
%A VM will generally stay on the cluster where it was deployed during its whole
%life cycle.
%
The main problem with this approach is that, in the absence of cooperation between cluster
managers, VMs cannot be migrated from one cluster to another, which is especially annoying
if one cluster is overloaded.
%
Moreover, the super manager is not necessarily aware of the network topology and therefore
may not be able to interact efficiently with cluster managers if the latter are
distributed among several sites.
%
Furthermore, the super manager limits the scalability of this approach; to deal with this
issue, researchers have designed distributed approaches.

Many distributed approaches have been proposed to manage
VMs~\cite{barbagallo:lncs10,feller:cloudcom12,marzolla:wowmom11,mastroianni:europar11,rouzaudcornabas:vhpc10,yazir:cloud10}.
%
Some of them are limited in terms of scalability since they (i)~require a global view of
the infrastructure to take a decision~\cite{rouzaudcornabas:vhpc10,yazir:cloud10} and/or
(ii)~rely on a centralized service node that is not
fault-tolerant~\cite{mastroianni:europar11,yazir:cloud10}.
%
Some approaches lead to a huge number of
migrations~\cite{barbagallo:lncs10,mastroianni:europar11} without necessarily optimizing
the chosen scheduling criterion~\cite{barbagallo:lncs10}.
%
Moreover, none of these approaches have been designed to take account of the network
topology and therefore manage VMs efficiently in a multi-site deployment.

To summarize, a locality-aware distributed approach is required to (i)~avoid issues
related to scalability and single points of failures, and to (ii)~manage VMs efficiently
in heterogeneous network environments like those found in multi-site/multi-cluster deployments.
The work presented in this paper targets such a challenge. 

\section{Introduction}
Introduced few years ago \cite{greenberg:sigcomm09}, the new trend to deliver
cloud computing resources, in particular Infrastructure as a Service solutions,
consists in leveraging several infrastructures  distributed world-wide. If such
distributed cloud computing platforms deliver undeniable advantages to address
important challenges such as reliability, latency or even in somehow
jurisdiction concerns, most mechanisms that were previously used to operate
centralized IaaS platforms must be revisited to offer the same level of
transparency for the end-users.  Keeping such an objective in mind, the use of
P2P paradigm is strongly investigated. This is particularly true for instance
for scheduling algorithms in charge of assigning VMs on top of PMs according to
their effective needs (and reciprocally usages).  Indeed and although major
improvements have been done, centralized approaches \cite{hermenier:2013} are
neither scalable nor robust enough.  Hierarchical solutions
\cite{feller:ccgrid12} that can be seen as good candidates face important
limitations: First, finding an efficient partitioning of resources is a tedious task as
matching a hierarchical overlay on top of a distributed infrastructure is often not logical. 
Second, in addition to requiring complex failover mechanisms to
ensure leader/super peer crashed and network disconnections, hierarchical
structures have not been designed to react swiftly to physical topology
changes such as node apparition/removal and network performance degradations. 
Considering scalability as well as resiliency criteria of distributed cloud computing infrastructures, 
it is not surprising that P2P approaches have shown promising results to address the scheduling problem of VMs \cite{xxx,quesnel:2012,feller:cloudcom12}.
% TODO Europar paper Flavien ?
However, it is noteworthy that manipulating VMs at WAN level is definitely worse in terms of performance and network impacts than doing LANWide. 
Hence,  it is fundamental to consider intra-site operations \vs inter-site ones.

 In other words, dyn

 
 

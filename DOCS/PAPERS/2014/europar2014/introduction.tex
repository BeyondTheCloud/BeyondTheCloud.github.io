Introduced a few years ago~\cite{greenberg:sigcomm09}, the new trend to deliver Cloud
Computing resources, in particular Infrastructure as a Service~(IaaS) solutions, consists
in leveraging several infrastructures distributed world-wide. If such distributed Cloud
Computing platforms deliver undeniable advantages to address important challenges such as
reliability, latency or even in somehow jurisdiction concerns, most mechanisms that were
previously used to operate centralized IaaS platforms must be revisited to offer the same
level of transparency for the end-users.

Keeping such an objective in mind, the use of the P2P paradigm has to be strongly
investigated. This is particularly true for scheduling algorithms in charge
of assigning virtual machines~(VMs) on top of physical machines~(PMs) according to their effective needs (and reciprocally
usages), to preserve a good quality of service~(QoS). Indeed and although 
major improvements have been done, centralized
approaches~\cite{hermenier:2013} are neither scalable nor robust enough. Hierarchical
solutions~\cite{feller:ccgrid12} that can be seen as good candidates face important
limitations: First, finding an efficient partitioning of resources is a tedious task as
matching a hierarchical overlay network on top of a distributed infrastructure is often not
natural. Secondly, in addition to requiring complex failover mechanisms to face
crashed leader/super peer and network disconnections, hierarchical structures have not
been designed to react swiftly to physical topology changes such as node
apparitions/removals and network performance degradations. P2P algorithms allow to
address both concerns, \ie scalability as well as resiliency of infrastructures. However,
and despite promising approaches have been proposed to address the scheduling problem of
VMs in a P2P fashion~\cite{quesnel:cpe2012,feller:cloudcom12}, they are still facing
limitations coming from the overlay network they rely on. The approach proposed
in~\cite{quesnel:cpe2012} maps a ring overlay network on a distributed infrastructure which
prevents making any distinction between close nodes and distant ones. Similarly, the
approach described in~\cite{feller:cloudcom12}, while adopting an orthogonal, gossip-based approach,
still suffers from building a randomized overlay network, thus breaking the physical topology.

Considering that both the network latency and the bandwidth between peers have a strong
impact on the reactivity criteria of the scheduling problem, \emph{locality} properties of
peers should be considered to favor efficient VM operations. In other words, to reduce as
much as possible the time to switch from one schedule (\ie a mapping between VMs and PMs
running in the infrastructure) to another one, it is crucial to make cooperation first
between peers in the closest neighborhoods before contacting peers belonging to other sites.
Moreover, it is noteworthy that this notion of locality is dynamic, and varies over time according
to the network bandwidth/latency and disconnections.

The contribution of this paper is a new building block that enables to tackle the
\emph{locality} concern in distributed VM scheduling algorithms such as the two
aforementioned ones.
% TODO open the discussion in the conclusion: Although the advantage of the
% contribution has been illustrated by revisiting the DVMS proposal, we believe
% that the proposed building block can be used in several distributed
% mechanisms based on a cooperative approach. 
We estimate the locality through a cost function of the latency/bandwidth tuple between
peers in the network, thus enabling each peer to select its closest neighbors. We rely on
Vivaldi~\cite{dabek:2001:sigcomm04}, a simple decentralized protocol allowing to map a
network topology onto a logical space while preserving locality. On top of Vivaldi, a shortest path
construction, similar to the well-known Dijkstra algorithm, is performed each time there is
a need for cooperation between two nodes taking part in the schedule. We illustrate the
advantage of this new building block by changing the overlay network in the DVMS
proposal~\cite{quesnel:cpe2012} for it. We selected DVMS as we have a good expertise of it and
because it is, as far as we know, the only one that guarantees to find a solution if one exists~\cite{quesnel:ispa2013}.
                                                   
The remainder of this article is structured as follows. In Section~\ref{sec:back}, we
discuss some background regarding the DVMS proposal and the P2P technics to handle
the locality aspects. Section~\ref{sec:contribs} gives an overview of our proposal by
introducing the short path algorithm on top of Vivaldi and the way we integrate it into
DVMS. In Section~\ref{sec:exps}, we validate the proposal by analyzing its benefits with
respect to the previous version of DVMS by discussing experiments conducted on
Grid'5000. Related works are discussed in Section~\ref{sec:related}. Finally, we discuss
perspectives and conclude this article in Section~\ref{sec:con}.


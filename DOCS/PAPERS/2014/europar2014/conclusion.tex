
Cloud Computing has entered our everyday life at a very high speed and huge
scale. From classic high performance computing simulations to the management of
huge amounts of data coming from mobile devices and sensors, its impact can no
longer be minimized.
% While a lot of progress has already been made in Cloud
%technologies, there are several concerns that limit the complete adoption of
%the Cloud Computing paradigm.
While promoted during a long time, delivering cloud computing capabilities by
leveraging only few large-scale data centers do not enable to cope with the
demand of cloud resources anymore and a new model consisting in leveraging
several micro/nano data centers distributed WANWide is more and more
investigated.
The main challenge is, thus, to revisit most of the mechanisms that are common
to current IaaS management systems in order to leverage more decentralized
algorithms. Amongst the different contributions that have been proposed, a large number has focused on the scheduling issue of the VMs 
in order to achieve the scalability required but at the expense of the locality criteria. 
However manipulating VMs WANwide degrades significantly the performance as well as the quality of the service of the whole system. 

% 
%In a previous report~\cite{lebre:hal-00854204}, we outlined that, in addition
%to these concerns, the current model of UC is limited by intrinsic issues.
%Instead of following the current trend by trying to cope with existing
%platforms and network interfaces, we proposed to take a different direction by
%promoting the design of a system that will be efficient and sustainable at the
%same time, putting knowledge and intelligence directly into the network
%backbone itself. The innovative approach we introduced will definitely tackle
%and go beyond Cloud Computing limitations. Our objective is to pave the way for
%a new generation of Utility Computing that better matches the Internet
%structure by means of advanced operating mechanisms. By offering the
%possibility to tightly couple UC servers and network backbones throughout
%distinct sites and operate them remotely, the LUC OS technology may lead to
%major changes in the design of UC infrastructures as well as in their
%environmental impact. The internal mechanisms of the LUC OS should be topology
%dependent and resources efficient. The natural distribution of the nodes
%through the different points of presence should be an advantage, which allows
%to process a request according to its scale: Local requests should be computed
%locally, while large computations should benefit from a large number of nodes.

Hence, the first step toward such a highly distributed Cloud infrastructure is to take
into account this notion of locality between cloud computing resources.  In
this paper, we shew how such locality criteria can be considered by delivering a new building block
using  P2P algorithms and a vivaldi overlay connected to the DVMS proposal, an efficient and
flexible VMs scheduler.  Our first experiments over Grid'5000 show that,
connecting 4 differents sites and scheduling VMs over them, we can gain up to
66\% of inter-sites operations. 

Our future work will consist in ... 

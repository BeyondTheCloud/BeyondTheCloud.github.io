\begin{abstract}
The promotion of distributed cloud computing infrastructures as the next 
platform to deliver the utility computing paradigm, leads to new virtual machines
(VMs) scheduling algorithms leveraging peer to peer approaches.
Although these proposals considerably improve the scalability, leading to the
management of hundreds of thousands of VM upon thousands of physical machines
(PMs), they do not consider the network overhead introduced by multi-site
infrastructures. This overhead can have a dramatic impact on 
performances if there is no mechanism for favoring intra-site vs. inter-site manipulations.  
% %%%
%To reduce such an impact,  locality properties should be considered as a key
%element, e.g. PMs should collaborate first with their neighbourghood from the
%same geographical site before contacing remote ones.  As network
%bandwidth/latency fluctuate over time, using a static partitionning of the
%resources is not enough. 
%
%
%This paper introduces a new building block built on a vivaldi overlay that
%maximizes efficient collaborations between PMs. We combined this mechanism with
%DVMS , a large scale virtual machine scheduler and show its benefit by
%discussing several experiments performed on four distinct sites of the
%Grid'5000 tesbed. Thanks to our proposal and without changing the scheduling
%decision algorithm, the number of inter-site operations has been reduced by
%66%.  This result provides a glimpse of the promising future of locality
%properties to improve performance of massive distributed cloud platforms.
%
% %%%
%In addition, as network properties (bandwidth, latency, sites interconnection)
%fluctuate over time, working with a static partitioning of the resources cannot
%be considered.

This paper introduces a new building block designed over a vivaldi overlay which 
maximizes efficient collaborations between PMs.
%\eg PMs should collaborate first with their neighbourghood from the
%same geographical site before contacing remote ones. 
 We combined this mechanism with
DVMS, a large scale virtual machine scheduler and showed its benefit by
discussing several experiments performed on four distinct sites of the
Grid'5000 testbed. Thanks to our proposal and without changing the scheduling
decision algorithm, the number of inter-site operations has been reduced by
66\%.  This result provides a glimpse of the promising future of locality
properties to improve performance of massive distributed cloud platforms.
 
 
%Future works will focus on the use of such a locality aware mechanisms to revisit other 
%mechanisms in the context of distributed cloud computing management. 

\keywords{Cloud computing, locality, peer to peer, network overlay, vivaldi, DVMS, virtual machine scheduling}
\end{abstract}

\section{Experimentations}

\subsection{Implementation of DVMS}

A prototype of DVMS leveraging locality based overlay has been developed. The 
current version of DVMS are been developed over the \emph{peer actor} 
abstraction which provides network abstraction that enables the design network 
overlay agnostic algorithm that are . We have developed two different
overlay for the Peer Actor abstraction: Chord and a locality based overlay over
Vivaldi.

The strength of this software architecture is that to enable an algorithm (like
DVMS) to comply with a given network overlay, it only have to follow the Peer 
Actor API. This way, we were able to run DVMS over Chord or Vivaldi without any
modification in its source code.

\subsection{Grid5000' experiments}

\JP{Complete this section with more experimentation results.}

\subsubsection{Objectives}
The prototype has been tested with a various number of experiments conducted on
the Grid5000' testbed. The main objective of the experiments was to estimate
impact of locality on the performance of a distributed scheduling algorithm. 
A significant portion of the reconfiguration time is spent in live migration of
virtual machines, which depends of network parameters such as latency and
bandwidth. One way to improve performance of distributed scheduling algorithm is
is to promote collaboration between close ressources, which can be reach by 
maximising this ratio:

\[
	\frac{number\ of\ intrasite\ migrations}{number\ of\ migrations}
\]

\subsubsection{Experimental protocol}
For each experiment, we booked 40 compute servers spread on 4 geographical sites
and 1 service server. The compute servers were used to run virtual machines and
DVMS while the service node is used to stress several parameters of 
virtual machines.

Each compute node will host a number of virtual machines proportional to the 
number of CPU cores it has. In our case:

\[
	number\ of\ virtual\ machines\ =\ 1.3\ \times\ number\ of\ cores
\]

\subsubsection{Results}

The impact of locality on DVMS is significant: using a Vivaldi based network
overlay leads to an average number of 83\% of intrasite migrations while using 
a Chord based DVMS leads a ratio of 50\% of intrasite migrations, as depicted
in the following table:


\begin{tabular}{|c|c|c|}
  
  % <HEADER>
  \hline
  network overlay & \multicolumn{1}{|p{3cm}|}{\centering average number of intrasite migrations}  & \multicolumn{1}{|p{3cm}|}{ \centering average number of migrations}  \\
  % </HEADER>

  % <ROW 1> => Vivaldi data
  \hline
  Vivaldi & 83 & 100 \\
  % </ROW 1>

  % <ROW 2> => Chord data
  \hline
  Chord & 40 & 80 \\
  % </ROW 2>

  \hline
\end{tabular}


\section{Revisiting OpenStack: towards a massively distributed IaaS manager}


\subsection{OpenStack: a toolkit for building clouds}

OpenStack is a very popular project that aim at building an opensource IaaS
manager. Many of the biggest actors of Cloud computing (Red hat, IBM, Rackspace,
VMware, Cisco, ...) are contributing to this project, thus providing new 
features at a rapide pace. OpenStack enables the construction of public or 
private clouds.

A typical cloud deployed with OpenStack is composed of several services (nova, 
swift, Quantum, Glance, ...) following the "shared nothing architecure" 
principle, meaning that each service is independent and thus shares no state 
with other services.

In this way, inter-services collaboration is performed by exchanging message 
through an AMQP (Advanced Messsage Queuing Protocol) based bus: each service 
has its own queue, and it collaborates with others by sending messages to their 
queues. This offers the advantage of easily plugging additional components : 
when a default service becomes no more suitable with system's needs, it can be 
naively replaced by another custom service, as long as the newer service 
consummes messages destinated to the older one, and in turn produces messages 
expected by collaborators.

As OpenStack is becoming a standard and since it provides mechansims that are
suitable to build clouds in data-centers context, it can be considerated as a
cloudkit of choice for the LUC-OS. 












\subsection{Revisiting components that are suitable for a LUC-OS (Related work)}
\label{sub:sec:revisiting_openstack}


As developing a complete IaaS manager is an Herculean work, we think that 
revisiting existing mechanims is a good way to reach our goal, even if it means 
that some of these mechanisms may be completely replaced.

For the sake of clarity, mechanisms that we plan to revisit will be studied 
following the list of services depicted in section \ref{sub:sec:list_services} :
for each service, the suitable mechanisms that will be revisited are indicated 
and proposed modification are detailled:

\begin{description}

	\item [Compute manager] : Nova is the OpenStack service that is 
	responsible for coordinating and controlling the work of other services.
	As it is the angular stone of OpenStack, we plan to directly plug the 
	LUC-OS on this service. Nova service contains a sub-service that is
	responsible for scheduling virtual machines: nova-scheduler, which performs
	static scheduling. This means that the scheduler will decide on which
	server the virtual machine will run once and for all. In section 
	\ref{sub:sec:integration_dvms} we discuss the replacement of 
	nova-scheduler by DVMS.


	\item [Administrative manager] : KeyStone and Horizon are two services
	that are respectively responsibles for managing identity/authentication
	and providing a web user interface to users that want to interact with
	OpenStack. As the LUC-OS will store every piece of information in a DHT: we
	propose to make KeyStone and Horizon pick data from this DHT. Furthermore 
	the LUC-OS and OpenStack does not exactly share the same semantics : for
	example the LUC-OS manipulates virtual environments which does not
	exactly exist in OpenStack. Revisiting Horizon service through minor
	modifications would enable the integration of this concept in OpenStack.

	\item [Storage manager] : Storage in OpenStack can be divided in two
	service: Swift manages objects (files) while Glance manages virtual machines
	disk images. Glance's images can be stored in Swift or in any other storage
	service. Recent solutions like VMTorrent \cite{reich:2012} have been
	developed to limit network overhead, for example during simultaneous 
	loading of images at VMs startup. For a first prototype we propose to 
	leverage Glance and Swift, and if it becomes no more suitable for a
	massive infrastucture, we propose to integrate VMTorrent in Glance and 
	to use it.

	\item [Network manager] : Neutron is the service that manages networking
	in OpenStack: it is used to create virtual networks (VLANs) that
	interconnect virtual machines. Neutron supports plugins : it is possible
	to use a virtual multilayer switch like Open vSwitch \cite{pfaff:2009}.
	In the case this plugin does not meet our needs, we propose to 
	leverage a software defined network (SDN) solution like Mininet.
	\cite{lantz:2010} or Vin.

\end{description}
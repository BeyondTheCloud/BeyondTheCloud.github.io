\section{Introduction}
\label{sec:intro} 


% - success of cloud computing.
% - ever-growing demand of computing resources (CR) -> production of CR.
% - economy of scale -> CR production is concentrated in mega datacenters.
The success of Cloud Computing has driven the advent of Utility Computing (UC): 
cloud providers have build large infrastructures that statisfy the ever-growing 
demand for computing resources. To realize economy of scale, the production
of computing resources is concentrated in mega data centers (DCs) of 
ever-increasing size with dedicated electrical and cooling systems.

% - mega DCs -> critical needs in electricity and cooling.
% - mega DCs in region with abundant and cheap electricity supply
% - mega DCs in region with free cooling. 
Furthermore, as mega DCs have critical needs in electricity and cooling, the
number of physical resources that one DC can host is limited by the capacity of
its energy supply and cooling system. The current trend is toward building DCs
in regions with abundant and affordable electricity supply or in regions close
to the polar circle to leverage free cooling techniques. 

% - ever-increasing DCs size 
%   -> more concentration of production (general) 
%     -> problems:
%        * fault-tolerance (disasters).
However this ever-increasing DCs size is a problem as it accentuates the 
concentration of the production of computing resources in a same geographical
area, thus leading to fault tolerance issues: when a disaster occurs 
connectivity to computing ressources cannot be guaranted.

% - alternative: deconcentration of computing resources.
% - federation of several clouds is a first is a solution for:
%      - fault tolerance.
%      - energy requirements.
As an alternative to this concentration of computing resources, federations of
clouds propose to extend resources available on one Cloud with those of another
one. One positive effect of federation of clouds is that it enables the 
decentralization of computings resources, thus solving the problems of 
connectivity in case of disasters, and dividing the electrical and cooling 
effort on several geographical sites.

% - mega DCs or federation of clouds -> data/apps far from users.
%   -> network overhead
% - IEEE report: network traffic has been doubling every years
% - example: CDNs decentralize the hosting of static resources.
However federations of clouds and mega DCs present the drawback of hosting 
computing resources far from end users leading to a network overhead. A recent 
IEEE report \cite{ieeenetreport:2012} shows that network traffic continues to 
double roughly every year. Consequently, a model bringing computing resources 
closer to the end-users would minimize the energy impact and save bandwidth. An
example of such model is Content Delivery Networks (CDNs) in web hosting : 
static resources like images and web scripts are duplicated on servers located
all over the world, enabling their distribution to end users with low latency 
and high bandwidth by delivering from the closest server.
%This enables large websites to 
%dedicate DCs to high value computing like dynamic content generation.

% - apply the CDN model to the cloud.
% - We propose: instead of concentrating production of computing resources:
%    * leverage the concept of Micro/nano DC geographically spread.
% - Operate these micro DCs with a cloud OS: instead of several Clouds OS that
%   only use remote clouds, we propose a single Cloud OS that operate all of 
%   them in a distributed manner.
In keeping with the decentralization of resource delivering as allowed by models
like CDNs, we propose to study a model where the production of computing 
resources is deconcentrated: leveraging the concept of geographically spread 
micro/nano DCs \cite{greenberg:2008}, we propose to directly locate them in ISP 
points of presence, thereby benefiting from high connectivity (low latency and
high bandwidth) with end users. These micro DCs will be operated by a Cloud
Operating System (Cloud OS): unlike federations of cloud where each cloud is
operated by its own system which uses rather than operate other Clouds, we 
propose to design a Cloud OS that will operate many micro DCs in a fully 
distributed manner.

% - Geographicaly spread DCs -> need to take into account locality properties.
% - We propose: a Cloud OS that takes into account locality properties.
% - This system will be build on top of distributed mechanisms such as DHT and
%   advanced distributed models (p2p).
% - LUC-OS: many micro DCs operated by a single system.
% - The LUC-OS will be very interesting for ISPs: 
%     backbone -> complete UC resources
To efficiently operate a massively distributed cloud build on top of close to 
end users micro DCs, we deeply think that a Cloud OS needs to takes into account 
several measurements (latency, bandwidth, ...) aggregated as locality 
properties. Indeed, a Cloud OS aware of these locality properties would 
efficiently organize collaborations between computing resources, thus improving 
the overall reactivity. Hence we propose the Locality based Utility Computing 
Operating System (LUC-OS), an advanced Cloud OS that will operate massively 
distributed clouds by leveraging locality properties and a peer to peer 
architecture to efficiently work in a fully distributed manner. To adress 
scalability and fault tolerance concerns, the LUC-OS will be build on top of
advanced distributed mechanisms like overlay networks, distributed hashtables 
and actor model. The LUC-OS will enable the unification of many UC resources 
distributed on distinct sites, to be operated through a single system. It would 
enable Internet service providers (ISPs) and other institutions in charge of 
operating a network backbone to build an extreme-scale LUC infrastructure with a 
limited additional cost. Instead of redeploying a complete installation, they 
will be able to leverage IT resources and specific devices such as computer room
air conditioning units, inverters or redundant power supply.



The remainder of this article is structured as follows. Section 2 discusses a 
reference architecture for clouds proposed by \cite{moreno2012iaas} by 
identifying challenges that need to be solved for building massively distributed 
clouds over geographically spread micro DCs. Section 3 gives an overview of the 
LUC OS design proposal that meet requirements from Section 2. In Section 4, 
mechanisms that we will revisited to build the LUC OS are detailed. Finally, we discuss perspectives and conclude this article in Section 5.
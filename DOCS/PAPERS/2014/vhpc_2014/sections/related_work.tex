\section{Related Work}
% - Current IaaS are completely/partially designed to work in a centralized way.
% - P2P1: -> Snooze, a hierarchical solution (complex).
% - P2P2: -> P2PCS (single type of node).
% - Cloud@Home: -> Volunteer computing + Cloud computing.
During the last five years, open-source IaaS managers like OpenStack have 
emerged. These managers are often used with centralized architectures, where 
each Cloud is operated by some service nodes. As these systems do not provide 
self-healing mechanisms, a failure of the service nodes located on a same site 
are SPOFs. As an alternative, authors of \cite{Babaoglu} have proposed to 
leverage the peer to peer paradigm by organizing large systems on top of 
self-organizing middleware.  

Thus several attempts have been made to distribute the functionning of Cloud 
managers: \emph{Snooze} \cite{feller:snooze2012} is 
a system that works in a fully distributed manner over a hierarchical 
structure. Despite validation with large-scale experiments, this hierarchical 
structure presents the drawback of increasing the complexity of the 
architecture: three kinds of nodes (local controller, group manager, group 
leader) constitutes the system. In \cite{babaoglu:design2012} authors proposed 
\emph{P2PCS}, a fully decentralized cloud system build on top of a flat 
unstructured network overlay, where each constituting node is identical to 
others. However, nodes are dedicated to only one user (slicing), \emph{P2PCS} 
doesn't reuse existing IaaS mechanisms and it has not been validated on large 
scale infrastructure.

Finally, we mention that in \cite{cunsolo:cloud2009} authors proposed 
\emph{Cloud@Home}, an hybrid system that bridges \emph{Cloud computing} and 
\emph{Volunteer computing}. This solution is partially centralized: while 
production of computing resources is distributed on member's computers, 
system's controllers constitutes SPOFs.
\section{Reference architecture for IaaS clouds}
\label{sec:moreno}

% - Designing LUC-OS -> complex (Cloud : a lot of aspects).
% - study existing Cloud architecture -> anticipation of these problems.
%    -> prevent from missing important points.
% - This will minimize conception effort -> maximize time spend on prototype.
% - This section -> highlight existing architectural principles (in order to use
%    them)
Designing an architecture for the LUC-OS is a complex task: it has to manage all
the very diverse aspects of a Cloud infrastructure. A good way to anticipate 
them and to prevent from missing any vital mechanisms is to study how the 
architecture of existing cloud solutions have been organized. This section 
targets to highlight existing architectural principles used by current Cloud 
managers, in order to make the LUC-OS leverage them.


% - Cloud Computing -> like electricity market.
% - Big fishes have emerged -> their APIs have become defacto standard.
% - emerging market -> yet no ISO norm, but standardization is coming.
% - Despite recent efforts, "data lock-in" still exists.
% - This lack of standardization is a drag on Cloud computing adoption.

% Cloud computing is a model where companies consume computing resources produced 
% by external cloud providers. This model has lead to a structure similar to the 
% electricity market: in the same way electrical providers have build huge 
% infrastructures for generating and transporting energy, Cloud providers now 
% leverage data-centers that can host more than tens of thousands of servers. As 
% this is an emerging industry, normalization is still at an early stage: several 
% ISO norms have been released \cite{ISO:2009:DAPS} and many consortiums have been
% created to promote the use of standards by Cloud Computing actors 
% \cite{marston:cloud2011}. However, this market is dominated by few actors 
% (Amazon, Google, Microsoft, Rackspace, ...) that exploit proprietary 
% technologies, thus standardization has been delayed and de facto standards have 
% emerged. Furthermore, despite recent efforts made to avoid "data lock-in", 
% migration from one Cloud provider to another one is still difficult to achieve. 
% All these factors are a drag on the adoption of Cloud computing.

% - To enable companies to manage their data in the way they want 
%   -> private clouds.
% - OpenSource IaaS manager have been developed (OpenStack, Open Nebula
%   ,...) to host private clouds.
% - Some studies compared them:
%    * They manipulate the same concepts (VMs, Networks, Storages, ...).
%    * They have architectural differences (CloudStack -> monolithic, OpenStack 
%      -> components, Eucalyptus -> AWS).
To keep the control on their data, organizations can leverage internally hosted 
Cloud infrastructure : private Clouds enable the utilization of the Cloud 
computing paradigm without depending to a Cloud provider. For this purpose, 
open-source IaaS managers have emerged: several of them have become popular with
large communities of developers and are used by large actors. Recent studies 
\cite{peng:2009} have compared these projects and it appears that while they 
almost manipulate same concepts (VMs, Networks, Storages, ...), they have taken
different orientations in their development: Eucalyptus focuses on building an 
infrastructure that shares the same API as Amazon AWS, while projects like 
OpenStack and CloudStack have targeted the delivery of a complete IaaS that can
operate large cloud infrastructures by using its own tools. This leads to an 
architectural heterogeneity : CloudStack’s development has been oriented towards
a monolithic architecture, while OpenStack is composed of loosely-coupled 
message passing managers and Eucalyptus follows the Amazon AWS services’
architecture.

% - Architectural differences -> Moreno's reference architecture.
% - several managers.
To cope with these architectural differences, Moreno et al. have proposed a 
reference architecture \cite{moreno2012iaas} for IaaS clouds. This architecture 
covers services that are needed for building a Cloud OS : each aspect of the 
cloud is supervised by a specific manager. The following listing gives a 
description of principal services that authors expect in a Cloud OS:
\begin{description}
	
	\item [Virtual machines manager] is the most important part of the Cloud OS:
	it manages VMs' cycle of life (configuration, deployment, 
	migration, suspension, resume, shut down and sometimes migrations). It is 
	also responsible for preserving service-level agreements (SLAs). Examples of
	such manager can be found in OpenStack's Nova service and CloudStack's 
	Kernel component.

	\item [Scheduler] decides which server will host a newly created virtual 
	machine (static scheduling). To preserve a good quality of service (QoS), 
	overloaded servers may migrate virtual machines on underloaded servers 
	(dynamic scheduling).

	\item [Image manager] is in charge of virtual machines' template files (VM 
	image). As users have their own images with specific systems or 
	configurations, this manager must be able to handle a large number of VM
	images in a distributed manner.

	\item [Network manager] provides connectivity to the infrastructure: virtual
	networks for VMs, external access for users. The current trend is towards 
	the	virtualisation of networking devices through the use of Software
	Defined Networks (SDN). This manager must enable flexible networking 
	deployment without losing performance (bandwidth and latency).

	\item [Storage manager] provides persistent storage facilities to virtual 
	machines. As physical single storage device might be a source of failure,
	factors like distribution and redundancy must be anticipated.

	\item [Administrative tools] provides tools and user interfaces to perform
	administration tasks. This is the components that enables users access to
	the Cloud OS internals.

	\item [Information manager, Accounting/auditing ] collects raw monitoring 
	data of the infrastructure and produce value-added from the raw data 
	respectively. As authors mentioned it, both are very important for billing
	process.

\end{description}

The other managers, not included in the preceding listing, are either taking 
place at different level in the Cloud Computing hierarchy (\emph{Service 
manager} is likely to be at PaaS level) or either too advanced for a first Cloud
OS prototype (\emph{Cloud Interfaces} exposes a cloud API to external users; 
\emph{Federation manager} enables the access to remote Clouds.) or is of little 
scientific interest for the Cloud OS (\emph{Information manager} gather VMs 
state information, as many tools meet this need, the Cloud OS will simply 
reuse them).

% - We want to develop a Cloud OS.
% - Develop from scratch: impossible (herculean work) -> maximize the reuse of
%   existing components.
% - We propose to leverage the reference architecture proposed by Moreno.
% - In order to develop a prototype, we focus of vital components.
We target the development of the LUC-OS : a fully distributed Cloud OS that can 
provides essential features proposed by current opens source IaaS managers. As
developing a complete Cloud OS is an herculean work, trying to maximize the 
reuse of existing works is a way to minimize both design and implementation 
efforts. Indeed, our first objective is to leverage the best practices 
established by the analysis of today's IaaS managers. We think that leveraging
this reference architecture is the most obvious way to cope with this first
objective.

% - Section 2: presentation of a reference architecture with challenges 
%   overview.
% - generic architecture -> can be applied to centralized, partially
%   distributed (federation of clouds) and fully distributed (LUC-OS) Cloud OS.
% - LUC-OS design is different of a classical IaaS manager (Controller ->
%   distributed share no state algorithm, DataBase -> DHT, Distributed FS -> p2p
%   FS), this section will expose the development methodology.
This Section exposed a reference architecture for building a Cloud OS, which 
have been proposed by \cite{moreno2012iaas}: this architecture is generic in the
sense that it provides a model where each aspect of the Cloud OS is managed by a
dedicated service. As a consequence, this model can be used as well for 
centralized clouds, partially distributed clouds (federations of clouds) and 
fully distributed clouds. The LUC-OS targets a fully distributed functioning, 
which is not the principal aim of current open-source Cloud managers, and thus
requires a significantly different architecture oriented towards the management 
of massively distributed clouds.

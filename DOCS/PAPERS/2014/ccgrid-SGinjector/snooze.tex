\subsection{Snooze}
\label{subsec:snooze}

\AL[MS]{1.5 pages}

We now present the Snooze framework for VM
management~\cite{feller:ccgrid12} as a third case study of how
to implement and simulate advanced distributed algorithms.

\subsubsection{Snooze architecture and  algorithm}

We first briefly present the Snooze architecture summarizing its
original presentation~\cite{feller:ccgrid12} and providing some
aditional info~\cite{snoozedev14}.

\begin{itemize}
  \item GLs, GMs, LCs
  \item Correct execution
  \item Fault behavior
\end{itemize}

\subsubsection{Simulation with Simgrid}

Snooze can be simulated using our model and tool support in a direct
and natural manner.

\begin{itemize}
  \item Representation of GLs, GMs, LCs
  \item Multicast node
  \item Multi-threading
  \item Reaction to faults
\end{itemize}

\subsubsection{Variants}

It is also possible to quite simply implement and simulate variants of
the Snooze model that we have simulated. These variants represent
interesting non-trivial evolutions of the original Snooze model that
have not been explored as part of Snooze. Concretely, we present the
following evolutions:

\begin{itemize}
  \item \textbf{Periodic vs.\ reactive scheduling:} 
  \item \textbf{Load balancing:} Snooze's implementation assigns LCs
    in a round-robin fashion. We can simply implement simulate other
    asignment schemes, e.g., load balancing schemes.
  \item \textbf{GMs rejoin on GM entry:} The Snooze paper~\cite{frm12a}
    states that all GMs should rejoin if a new GMs enters the system
    (perhaps for load balancing). This is implemented in the Snooze
    simulation. However, this strategy makes the system instable,
    because of frequent joins of GMs that are not available during a
    relatively long time. Furthermore, this entails joins of many
    LCs. Overall this strategy seems unreasonable.
\end{itemize}



%%% Local Variables: 
%%% mode: latex
%%% TeX-master: "main"
%%% End: 

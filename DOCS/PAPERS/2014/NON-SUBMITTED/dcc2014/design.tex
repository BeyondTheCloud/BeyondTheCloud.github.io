\section{Designing a massively distributed Cloud}
\label{sec:design}


\subsection{Toolkit for IaaS}

% \begin{itemize}

% 	\item A toolkit is a building block that can be used for the construction of
% 	systems (generic definition of a software toolkit).

% 	\item The objective of a toolkit is to provide "state of the art" solutions
% 	to known problems. It enables the focus on "Top level" works.

% 	\item It provides a set of components, which once assembled constitute an 
% 	operational system.

% 	\item Recent studies of "state of art IaaS systems" (OpenStack, Cloudstack,
% 	OpenNebula, ...) showed that they were constructed over same concepts. It 
% 	enables the design of IaaS toolkit.

% 	\item The massively distributed IaaS toolkit will provides "state of the 
% 	arts" mechanism to solve both scalability and locality points.

% 	\item The toolkit will have to integrate well on existing systems: we 
% 	propose to leverage OpenStack project.

% \end{itemize}

A software toolkit is a set of software building blocks that includes state of 
the art mechanisms for known problems. A toolkit comes with an API (Application
Programming Interface) which is the specifications that ones must follow to
correctly use provided mechanisms. The goal of a toolkit is to enable developers
to focus on the creation of higher level mechanisms, thus speeding up the 
development time.

An IaaS toolkit should be delivered with a set of default high level mechanisms 
whose assembly results in an basic operational IaaS system. In the case where 
one of the default constituting mechanisms would not be sufficient, it should be
redeveloped by leveraging the toolkit's low level mechanisms.

Recent studies have showed that state of the art IaaS manager \cite{peng:2009}
were constructed over the same concepts. Furthermore a reference architecture 
for IaaS manager has been described in \cite{moreno2012iaas} enabling the design
of an IaaS toolkit. Besides the reference architecture, an IaaS toolkit should
provides some mechanisms that enables to solve the scalability problem.

To maximize the chance of being reused by a large community, an IaaS toolkit 
should enable an easy integration with one or several existing IaaS cloud
managers. In our case we have chosen to leverage the OpenStack project: as a 
result the mechanisms developped with the toolkit will integrate well with 
existing clouds that are based on OpenStack.



\subsection{Massively distributed cloud}

% \begin{itemize}

% 	\item A massively distributed cloud targets management of thousand of hosts 
% 	around a wide territory.

% 	\item This scale order is currently reached by file sharing systems like 
% 	bittorrent.

% 	\item At this scale, failure becomes the norm.

% 	\item Recent works propose to leverage on peer to peer overlay.

% 	\item Some peer to peer overlays can take advantage of locality. It enables
% 	to build systems that can take into account network bandwidth and latency.

% 	\item We propose to leverage on locality based peer to peer mechanisms to 
% 	reach an high scalability IaaS.

% \end{itemize}

Cloud providers concentrate the production of computing resources in 
data-centers that contains tens of thousand of servers, enabling IaaS mechanisms
to take advantage of fast network with extremely low latency. However this ever 
increasing data-centers size has become a problem, as many data-centers require 
dedicated electrical and cooling infrastructure. As an alternative to 
concentrating the production of computing resources, we propose to study a model
where this production is deconcentrated.

Leveraging the concept of micro data-centers proposed by \cite{greenberg:2008},
we suggest to build a cloud operating system that will run in a distributed
manner over a set a small data-centers geographically spread. This cloud 
operating system will have to reach high scalability criteria: managing 
thousands of servers used by hundreds of users. Popular peer to peer file
sharing systems already work at this scale order: bittorrent clients enable 
hundreds of thousands of users to share millions of file spread over the 
internet. If we disregard trackers, this protocol is totally decentralized with 
no single point of failure (SPOF). That is why we propose to learn from peer to 
peer file sharing experience, in order to build massively distributed clouds.

As at this scale failure becomes the norm rather the exception, it is vital to
take into account fault tolerance in the early stages of design, by leveraging a
peer to peer overlay network. As we think that working in a massively
distributed context require to deal with network parameters like 
latency and bandwidth usage, collaboration between the constituting nodes of the
system should be organized in a "network aware" manner. For instance, the 
Vivaldi algorithm \cite{dabek:2004:vivaldi} provides a dynamic coordinate system
that can be used to introduce locality properties inside a distributed system, 
thus building low latency collaborations.

We assume that the architecture of massively distributed clouds should be 
organized arround the same principles that rule communautary file sharing 
systems. That is why we propose in section \ref{sec:architecture} an 
architecture that will be build on top of peer to peer principles and 
articulated arround a locality based overlay network. To meet the high 
scalability criteria, some of IaaS mechanisms should be revisited to improve 
reactivity of inter-servers collaboration by leveraging locality properties.




%%%
\section{Introduction\label{sec:intro}}

% - success of cloud computing.
% - ever-growing demand of computing resources (CR) -> production of CR.
% - economy of scale -> CR production is concentrated in mega datacenters.
% - mega DCs -> critical needs in electricity and cooling.
% - mega DCs in region with abundant and cheap electricity supply
% - mega DCs in region with free cooling.
To satisfy the escalating demand for Cloud Computing (CC) resources while realizing an economy of scale, the production of computing resources is concentrated in mega data centers (DCs) of ever-increasing size, where the number of physical resources that
one DC can host is limited by the capacity of its energy supply and its cooling system. To meet these critical needs in terms of energy supply and
cooling, the current trend is toward building DCs in regions with abundant and affordable electricity supplies or taking advantage of free cooling
techniques available in regions close to the polar circle~\cite{greenpeace:2013}.

% - ever-increasing DCs size
%   -> more concentration of production (general)
%     -> problems:
%        * fault-tolerance (disasters).
% - alternative: deconcentration of computing resources.
% - federation of several clouds is a first is a solution for:
%      - fault tolerance.
%      - energy requirements.
% - mega DCs or federation of clouds -> data/apps far from users.
%   -> network overhead
% - IEEE report: network traffic has been doubling every years
% - example: CDNs decentralize the hosting of static resources.

%% However,  concentrating  Mega-DCs  in  only  few  attractive  places  implies
%% different  issues such  as  exceptional  phenomenon\footnote{\small On  March
%% 2014, a large crack has been found  in the Wanapum Dame leading to emmergency
%% procedures. This  hydrolic plan  supports the utility  power supply  to major
%% data centers in central Washington.}: when a disaster happens in these areas,
%% the  connectivity to  CC  resources  cannot not  be  guaranteed, thus  having
%% dramatic  consequences  for   IT  services  that  were   depending  on  these
%% ressources.  Second, due  to  the geographical  dispersal  of users,  hosting
%% computing resources in a few locations  leads to a network overheads to reach
%% DCs.  Such overhead  can  be a  cause  of preventing  the  adoption of  cloud
%% computing by  latency critical applications  such as mobile computing  or big
%% data ones.

However, concentrating Mega-DCs in only few attractive places implies different issues. First, a disaster\footnote{On March 2014, a large crack has
  been found in the Wanapum Dame leading to emmergency procedures. This hydrolic plan supports the utility power supply to major data centers in
  central Washington.} in these areas would be dramatic for IT services the DCs host as the connectivity to CC resources would not be
guaranteed. Second, in addition to jurisdiction concerns, hosting computing resources in a few locations leads to useless network overheads to reach
each DC. Such overheads can prevent the adoption of Cloud Computing by several kind of applications such as mobile computing or Big Data ones.

% - We propose: instead of concentrating production of computing resources:
%    * leverage the concept of Micro/nano DC geographically spread.
% - Operate these micro DCs with a cloud OS: instead of several Clouds OS that
%   only use remote clouds, we propose a single Cloud OS that operate all of
%   them in a distributed manner.
% - Classical models (centralized, federations) -> not good (SPOF, does not
%  operate efficiently).

The concept of micro/nano DCs at the edge of the backbone~\cite{greenberg:2008} is a promising solution for the aforementioned concerns. However,
operating multiple small DCs breaks somehow the idea of mutualization in terms of physical resources and administration simplicity, making this
approach questionable. One way to enhance mutualization is to leverage existing network centers, starting from the core nodes of the network backbone
to the different network access points (\aka. PoPs – Points of Presence) in charge of interconnecting public and private institutions. By hosting
micro/nano DCs in PoPs, it becomes possible to mutualize resources that are mandatory to operate network/data centers while delivering widely
distributed CC platforms better suited to cope with disasters and to match the geographical dispersal of users. % and their needs.
A preliminary study has established the fundamentals of such an \emph{in-network distributed cloud} referred by the authors as the
\emph{Locality-Based Utility Computing} (LUC) concept~\cite{lebre:beyond2013}. However, the question of how operating such an infrastructure still
remains. Indeed, at this level of distribution, latency and fault tolerance become primary concerns, and collaboration between components that are
hosted on different location must be organized wisely.

%
%
% - Developing from scratch: herculean work.
% - We should maximize the reuse of existing tools/concepts.
% - To do so: 1) draw architectural summary; 2) identify challenges for the
%   LUC-OS 3) instantiate it over OpenStack.
%
In this report, we propose to discuss some key-elements that motivate our choices to design and implement the \emph{LUC Operating System} (LUC-OS), a
system in charge of turning a LUC infrastructure into a collection of abstracted computing facilities that are as convenient to administrate and that
can be used in the same way as existing Infrastructure-as-a-Service (IaaS) managers~\cite{cloudstack,opennebula,openstack}. We explain, in particular,
why federated approaches~\cite{buyya:intercloud} are not satisfactory enough to operate a LUC infrastructure and why designing a fully distributed
system makes sense. Moreover, we describe the fundamental capabilities the LUC OS should deliver. Because they are similar to those provided by
existing IaaS managers and because technically speaking it would be a non-sense to develop the system from scratch, we chose to instanciate the LUC OS
concept on top of the OpenStack solution~\cite{openstack}.

The main contribution is a proof of concept of the \emph{Nova} service (the OpenStack compute element) that has been distributed on top of a
decentralized key/value store (KVS). By such a mean, it becomes possible to efficiently operate several geographical sites by a single OpenStack
system. The correct functioning of this proof of concept has been validated via several experiments performed on top of Grid'5000~\cite{grid5000}. In
addition to tackling both the scalability and distribution issues of the SQL database, our KVS proposal leads to promising performance. More than 80\%
of the API requests are performed faster than with the SQL backend without doing any modification in the \emph{Nova} code.
%For the last 20\%, we need
%to und will require to dive deeper into OpenStack to understand all possible
%performance degradations.

The remaining of the report is as follows. Section~\ref{sec:design} explains our design choices. Section~\ref{sec:leveraging-openstack} describes
OpenStack and how we revised it. The validation of our prototype focusing on the Nova service is presented in Section~\ref{sec:eval}. In
Section~\ref{sec:ongoing_work}, we present the ongoing works towards a complete LUC~OS leveraging the OpenStack ecosystem. Finally
Section~\ref{sec:conclusion} concludes and discusses future research and development actions.

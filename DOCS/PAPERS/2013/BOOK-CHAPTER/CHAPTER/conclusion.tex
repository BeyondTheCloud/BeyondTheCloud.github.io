%%%
\section{Conclusion \label{sec:conclusion}}


Cloud Computing has entered our everyday life at a very high speed and huge
scale. From classic high performance computing simulations to the management of
huge amounts of data coming from mobile devices and sensors, its impact can no
longer be minimized. While a lot of progress has already been made in Cloud
technologies, there are several concerns that limit the complete adoption of
the Cloud Computing paradigm. %Amongst others, resource management as well as
%scale and heterogeneity of modern IT environments are two of the research
%issues highlighted in the recent Advance in Clouds report\footnote{\url{http://cordis.europa.eu/fp7/ict/ssai/docs/future-cc-2may-finalreport-experts.pdf}}.

In this paper, we outlined that, in addition to these concerns, the current model
of UC is limited by intrinsic issues.  Instead of following the current trend by trying to
cope with existing platforms and network interfaces, we proposed to take a
different direction by promoting the design of a system that will be efficient and sustainable at the same time, putting knowledge and intelligence
directly into the network backbone itself.

The innovative approach we introduced will definitely tackle and go beyond
Cloud Computing limitations. Our objective is to pave the way for a new
generation of Utility Computing that better matches the Internet structure
by means of advanced operating mechanisms.  By offering the possibility to
tightly couple UC servers and network backbones throughout distinct sites and
operate them remotely, the LUC OS technology may lead to major changes in the
design of UC infrastructures as well as in their environmental impact.  The
internal mechanisms of the LUC OS should be topology dependent and resources
efficient. The natural distribution of the nodes through the different points
of presence should be an advantage, which allows to process a request according
to its scale: Local requests should be computed locally, while large computations
should benefit from a large number of nodes.

Finally, we believe that LUC investigations may contribute to fill the gap
between the distributed computing community and the network one. This
connection between these two communities has already started with the different
activities around Software-defined Networking and Network as a Service. This may
result in the long view in a new community dealing with UC challenges where
network and computational concerns are fully integrated. Such a new community
may leverage the background of both areas to propose new systems that are more
suitable to accommodate the needs of our modern societies.

We are of course aware that the design of a complete LUC OS and its adoption by
companies and network providers require several big changes in the way UC
infrastructures are managed and wide area networks are operated. However we are
convinced that such an approach will pave the way towards highly efficient as
well as sustainable UC infrastructures, coping with heterogeneity, scale, and
faults. 

%%%


%%%
\section{Background\label{sec:background}}

Several generations of UC infrastructures have been proposed and still
co-exist~\cite{foster:2011}. However, neither Desktop, nor Grid, nor Cloud Computing
platforms provide a satisfying UC model.  Contrary to the current trend that promotes
large offshore centralized DCs as the UC platform of choice, we claim that the only way to
achieve sustainable and highly efficient UC services is to target a new infrastructure
that better matches the Internet structure.  Because it aims at gathering an unprecedented
amount of widely distributed computing resources into a single platform providing UC
services close to the end-users, a LUC infrastructure is fundamentally different from
existing ones.  Keeping in mind the aforementioned objectives, recycling UC resource
management solutions developed in the past is doomed to failure.

As previously mentioned, our vision significantly differs from hybrid Cloud Computing
solutions.  Although these research activities address important concerns related to the
use of federated cloud platforms such as the standardization of the interfaces for
supporting cooperation and resource sharing over Cloud federations, their propositions are
incremental improvements of the existing UC models. Recent hybrid and
cloud federation investigations are comparable in some ways to previous works that have
been done for Grids as the purpose of Grid middleware is to interact with each
resource management system composing the Grid
\cite{buyya:2010,rochwerger:2009,zhao:2012}. By taking into account network issues in
addition to traditional computing and storage concerns in Cloud Computing systems, the
European SAIL
project\footnote{\href{http://www.sail-project.eu}{\url{http://www.sail-project.eu}}} is
probably the one which targets the biggest advances in regard to previous works on
Grid systems. More concretely, this project investigates new network
technologies to
provide end-users of hybrid/federated Clouds the possibility to configure and
virtually operate the network backbone interconnecting the different sites they
use~\cite{sail:2012}.
%
More recently, the \emph{Fog Computing} concept has been proposed
as a promising solution to applications and services that
cannot be put into the cloud due to locality issues (mainly latency and
mobility concerns)~\cite{bonomi:2012}.  Although it might look similar to our vision as they
propose to extend the Cloud Computing paradigm to the edge of the network,
\emph{Fog Computing} does not target a unified system but rather proposes to
add a third party layer (\textit{i.e.} the \emph{Fog}) between cloud vendors and
end-users.
%
In our vision, UC resources (\textit{i.e.} Cloud Computing ones) should be repacked in
the different network hub of backbones and operated through a unified system, \textit{i.e.} the LUC OS.
%
As far as we know, the only system that investigated whether a
widely distributed infrastructure can be operated by a single system, was the
XtreemOS Project \cite{morin:2007}. Although this project was sharing some of
the goals of the LUC OS, it did not investigate how the geographical
distribution of resources can be leveraged to deliver more efficient and sustainable
UC infrastructures. 
%
%In addition to consider \emph{Locality} as a primary concern, 
%the novelty of the LUC OS proposal is a new way of designing systems to operate UC resources.
%Unlike existing research on distributed operating systems that consider processes as
%the basic object they manipulate, a LUC OS  will manipulate virtual
%machines. Virtualization technologies abstract hardware
%heterogeneity, and allow transparent deployment, preemption, and
%migration of VEs.
%% That is why we think that leveraging virtualization can have a deep impact on the way
%%distributed resources can be operated. 
%By dramatically increasing the flexibility of resource management,  virtualization will
%allow to leverage state-of-the-art results from other distributed
%systems areas such as autonomous and P2P/decentralized-based techniques. 

To sum up,  we argue for the design and the implementation of a distributed OS,
manipulating VEs instead of processes and considering locality as a primary concern. 
Referred to as a LUC Operating System, such
a system will include most of the mechanisms that are common to current UC
management systems~\cite{cloudstack,nimbus,opennebula,openstack,lowe:wiley11,moreno:2012}.  However,
each of them will have to be rethought in order to leverage P2P algorithms.
% that are mandatory 
%to meet the characteristics as well as the objectives of the targeted LUC platform. 
%
While largely unexplored for building operating systems,
P2P/decentralized mechanisms have the potential to achieve the scalability required
for LUC systems.
Using all these technologies for establishing the foundation mechanisms of
massive-scale distributed operating systems will be a major breakthrough from
current static, centralized or hierarchical management solutions.

%Finally, looking one step ahead and considering that large part of the Internet may be
%seen as computing services hosted by UC platforms, the question might
%be extended to : Will \discovery make possible to host/operate a large  part of
%the Internet by its internal structure itself?
%Similarly to the interconnection of academics and private networks and the use of the TCP/IP standard
%that resulted in the Internet, the deployment and the use of the \discovery system through the
%different networks might lead to a global UC platform, \textit{i.e.} \emph{the
%\discovery infrastructure}: A scalable and nearly infinite set of resources
%delivered by any computing facilities forming the Internet, starting from the
%larger hubs operated by ISPs, government and academic institutions to any idle
%resources that may be provided by end-users.
%\ftodo{AL}{Copy/Paste from the abstract}
%

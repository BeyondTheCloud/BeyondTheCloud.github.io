%%%
\vspace*{-.5cm}
\section{Overall Vision and Major Challenges\label{sec:challenges}}

Similarly to traditional operating systems~(OSes), a LUC OS will be composed of
many mechanisms. Trying to identify all of them and establishing how they
interact is an on-going work (see Section~\ref{sec:archi}). However,
%having in
%mind the goal of delivering a unified system in charge of operating a complex and diverse
%infrastructure, and transforming it into a LUC platform,
we have pointed out the following
key objectives to be considered when designing a LUC OS:

\begin{itemize} 
\item \textbf{Scalability:} A LUC OS must be able to manage hundreds of
  thousands of virtual machines~(VMs) running on thousands of 
  geographically distributed computing resources.  These resources are small or
  medium-sized computing facilities and may become highly volatile according to the network disconnections.  
\item \textbf{Reactivity:} To deal with the dynamicity of the infrastructure, a LUC OS
  should swiftly handle events that require performing particular
  operations, either on virtual or on physical resources. This has to be done with the
  objective of maximizing the system utilization while meeting the quality of service~(QoS) expectations of VEs. 
  Some examples of operations that should be performed as fast as possible include (i)~the reconfiguration
  of VEs over distributed resources, sometimes spread across wide area networks, or (ii)~the migration of VMs, 
  while preserving their active connections.
\item \textbf{Resiliency:} In addition to the inherent dynamicity of the
  infrastructure, failures and faults should be considered as the norm rather than the
exception at such a scale. The goal is therefore to transparently leverage the
underlying infrastructure redundancy to (i)~allow the LUC OS to keep
working despite node failures and network disconnections (LUC OS robustness) and to (ii)~provide
snapshotting as well as high availability mechanisms for VEs (VM robustness).
\item \textbf{Sustainability:} Although the LUC approach would reduce the energy
footprint of UC services by minimizing the cost of the network, 
it is important to go one
step further by considering energy aspects at each level of a LUC OS
and propose advanced mechanisms in charge of making an optimal usage of each source of energy. 
%Minimizing the energy footprint is a
%  transversal concern that has to be considered at each level of the
%  design of \discovery.
  To achieve such an objective, the LUC OS should take account of data related to the
  energy consumption of the VEs and the computing resources, as well as the environmental
  conditions (computer room air conditioning unit, location of the site, etc.).
\item \textbf{Security and Privacy:} Similarly to resiliency, security, and privacy issues
  affect the LUC OS itself and the VEs running on it. Regarding the LUC OS, the goals are
  to (i)~create trust relationships between different locations, (ii)~secure the
  peer-to-peer layers, (iii)~include security and privacy decision and enforcement points
  in the LUC OS and (iv)~make them collaborate through the secured peer-to-peer layers to
  provide end-to-end security and privacy.
%  at different layers and locations to provide a end-to-end and in-depth security enforcement.
  Regarding the VEs, users should be able to express their requirements in terms of
  security and privacy; these requirements would then be enforced by the LUC OS.

% \ftodo[AL/JP]{More on privacy? Only security is mentioned.}

\end{itemize}

In addition to the aforementioned objectives, working on a virtual infrastructure requires
to deal with the management of VM images. Managing VM images in a distributed way across a
wide area network~(WAN) is a real challenge that will require to adapt state-of-the-art
techniques related to replication and deduplication. Also, the LUC OS must take into
account VM images location, for instance (i)~to allocate the right resources to a VE or
(ii)~to prefetch VM images, to improve deployment performance or VM relocations.

Finally, one last scientific and
technical challenge is the lack of a global view of the infrastructure.  Maintaining a
global view would indeed limit the scalability of the LUC OS, which is inconsistent with
our objective to manage large-scale geographically distributed systems.  Therefore, we
claim that the LUC OS should rely on decentralized and autonomous mechanisms, that can
match and adapt to the volatile topology of the infrastructure.  Several decentralized
mechanisms are already used in production on large-scale systems; for instance, Amazon
relies on the Dynamo service~\cite{decandia:2007} to create distributed indexes and
recover from data inconsistencies; moreover, Facebook uses Cassandra~\cite{lakshman:2010},
a massive scale structured store that leverages peer-to-peer techniques.
%
%Among the numerous scientific and technical challenges that should be addressed, 
%the lack of a global view of the system introduces a lot
%of complexity. In order to tackle it while addressing the above-mentioned
%challenges, we claim that internal mechanisms of a LUC OS should be based
%on decentralized mechanisms specifically designed for it.
%% the latest contributions in distributed and
%% cooperative algorithms such as gossip-based approaches and self-* techniques.
%These techniques should provide mechanisms which are fully decentralized and
%autonomous, so to allow self-adapting control and monitoring of complex
%large-scale systems. Simple locality-based actions by each of the entities
%composing the system can lead to the global emergence of complex and
%sophisticated behaviors, such as the self-optimization of resource allocation,
%or the creation of decentralized directories. These techniques are starting to
%be used in well-known large systems. As an example, the Amazon website relies on
%its decentralized Dynamo service~\cite{decandia:2007} to create largely distributed indexes and recover from data
%inconsistencies. Facebook’s Cassandra massive scale structured
%store~\cite{lakshman:2010} also leverages P2P techniques for its core
%operation.
%
In a LUC OS, decentralized and self-organizing overlays will enable to maintain the
information about the current state of both virtual and physical resources, their
characteristics and availabilities. Such information is mandatory to build higher-level
mechanisms ensuring the correct execution of VEs throughout the whole infrastructure.
%However, it is worth noting that simultaneous local actions can lead to the global
%emergence of complex behaviors.


%%%
\section{Overall Vision and Major Challenges\label{sec:challenges}}

Similarly to traditional operating systems (OSes), a LUC OS will be composed of a
significant number of mechanisms. Trying to identify all of them and establishing how they
interact is an on-going work (see Section~\ref{sec:archi}). However, having in
mind the goal of delivering a unified system in charge of operating a complex and diverse
infrastructure, and transform it into a LUC platform, we have identified the following
objectives to be considered when designing a LUC OS:

\begin{itemize} 
\item Scalability: a LUC OS must be able to manage hundreds of
  thousands of virtual machines (VMs) running on thousands of 
  geographically distributed computing resources, including small and
  medium-sized computing facilities as well as any idle resource that their owner would make available. These resources might be
  highly volatile, especially if the LUC infrastructure allows to include resources hosted by
  end-users.
\item Reactivity: To deal with the dynamicity of the infrastructure, a LUC OS
  should swiftly handle events that require performing particular
  operations, either on virtual or on physical resources, with the
  objective of maximizing the system utilization while meeting QoS expectations of VEs. 
  Reconfiguring  VEs over distributed resources, sometimes spread across wide area networks, or moving VMs, 
  while preserving their active connections, are examples of operations that should be performed as fast as possible.
\item Resiliency: In addition to the inherent dynamicity of the
  infrastructure, failures and faults should be considered as the norm rather than the
exception at such a scale. The goal is therefore to transparently leverage the
underlying infrastructure redundancy to (i)~allow the LUC OS to keep
working despite node failures and network disconnections and to (ii)~provide
snapshotting as well as high availability mechanisms for VEs.

\ftodo[CT (+FQ)$\rightarrow$AL]{les points (i) et (ii) ne sont pas indépendants, j'ai
  l'impression que pour résoudre le point (i), on va utiliser ce que tu
  mentionnes dans le point (ii), du coup, c'est pas vraiment une énumération.}

\item Sustainability: Although the LUC approach natively reduces the energy
footprint of UC services by minimizing the network impact, 
\ftodo[FQ$\rightarrow$AL]{Ce n'est pas l'avis du reviewer de Middleware. Et nous n'avons à ma connaissance 
pas de données pour affirmer une telle chose}
it is important to go one
step further by considering energy aspects at each level of a LUC OS
and propose advanced mechanisms in charge of an optimal usage of each source of energy. 
%Minimizing the energy footprint is a
%  transversal concern that has to be considered at each level of the
%  design of \discovery.
 To achieve such an objective, data related to the energy
  consumption of the VEs  and the computing resources
  as well as the environmental conditions (computer room air conditioning unit, location of the
  site, etc.) should be taken into account by the system.
\item Security and Privacy: Similarly to resiliency, security issues affect the LUC OS itself and the VEs running on it.
For the LUC OS security, the goals are to (i) create trust relationships between
different locations, (ii) secure the P2P layers, 
(iii) include security decision and enforcement points in the LUC OS and (iv) make them collaborate through the secured P2P layers to provide a secured infrastructure.
%  at different layers and locations to provide a end-to-end and in-depth security enforcement.
For the VEs security, we need to provide users with a way to express their security requirements that will be enforced by collaborating 
several LUC OS security decision and enforcement points.

\ftodo[AL/JP]{More on privacy? Only security is mentioned.}

\end{itemize}

In addition to the aforementioned objectives, targeting a distributed system
where VM is the elementary granularity requires to deal with important issues
regarding the management of the VM images. Managing VM images in a distributed
way across a WAN is a real challenge that will require to adapt
state-of-the-art techniques of replication and deduplication. Also,
several mechanisms of a LUC OS must take into account VM images'
location, for instance to allocate the right resources to a VE  or to request
VM images prefetching to improve deployment performance or VM relocations.

Amongst the numerous scientific and technical challenges that should be addressed, 
the lack of a global view of the system introduces a lot
of complexity. In order to tackle it while addressing the above-mentioned
challenges, we claim that internal mechanisms of a LUC OS should be based
on decentralized mechanisms specifically designed for it.
% the latest contributions in distributed and
% cooperative algorithms such as gossip-based approaches and self-* techniques.
These techniques should provide mechanisms which are fully decentralized and
autonomous, so to allow self-adapting control and monitoring of complex
large-scale systems. Simple locality-based actions by each of the entities
composing the system can lead to the global emergence of complex and
sophisticated behaviors, such as the self-optimization of resource allocation,
or the creation of decentralized directories. These techniques are starting to
be used in well-known large systems. As an example, the Amazon website relies on
its decentralized Dynamo service~\cite{decandia:2007} to create largely distributed indexes and recover from data
inconsistencies. Facebook’s Cassandra massive scale structured
store~\cite{lakshman:2010} also leverages P2P techniques for its core
operation.
%
In a LUC OS, decentralized and self-organizing overlays will enable to maintain
the information about
the current state of both virtual and physical resources, their characteristics and
availabilities. Such information is mandatory to build higher-level mechanisms ensuring the correct execution of VEs throughout 
the whole infrastructure. 
 
% To ensure scalability as well as flexibility, they should be constructed and operated using gossip-based techniques. 
%Considerable research needs to be conducted to provide core operating systems services in a
%fully decentralized way and without centrally maintained comprehensive knowledge of system's resources


\subsection{Reliability Mechanisms}
%These mechanisms should ensure  reliability and high availability of the
%\discovery system despite the scale and dynamicity of the underlying physical
%infrastructure.   
Although we can expect that  the frequency of failures on LUC resources should be similar than 
in current UC platforms, it is noteworthy to mention that the expected mean time to repair failed equipments might be much higher 
since resources will be highly distributed. For these reasons, specific 
mechanisms should be designed to manage failures transparently
with a minimum downtime. 

%% It includes
%% mechanisms to ensure the high availability of the \discovery system
%% itself and mechanisms to allow executing VEs reliably.

Ensuring the high availability of the \discovery system requires being able to
autonomously relocate and restart any service on a healthy node in case of failure.
Moreover, a Cassandra-like framework~\cite{lakshman:2010} is required
to avoid losing or corrupting
information belonging to stateful services, 
since it provides a reliable and highly available back-end.

Regarding VEs reliability, a first level of fault tolerance can be
provided by leveraging periodical VM snapshotting capabilities.
In case of failure, a VE can be restarted from its latest snapshot.
Performing VM snapshotting in a large-scale,
heterogeneous, and widely spread environment is a challenging
task. However, we believe that adapting ideas that were recently proposed
in this field~\cite{nicolae:2011} would allow us to provide such
a feature.

Snapshotting is not enough for services that should be made highly
available, but a promising solution is to use VM
replication~\cite{Petrovic2012}. To implement VM replication in a WAN,
solutions to optimize synchronizations between
replicas~\cite{gerofi:2012,rajagopalan:2012} should be
investigated. Also, we think that a LUC infrastructure has a major advantage over
other UC platforms, since it is tightly coupled with the network
infrastructure. As such, we can expect \emph{low} latencies between
nodes, that would enable us to provide a strong consistency between
replicas while achieving acceptable response time for the replicated
services. 
%Note that the overlays presented in Section~\ref{ssec:p2p}
%for replicas localization and monitoring. Locating replicas based on a
%navigating scheme where at least one bridge is encountered, would be
%enough to ensure that they have a low probability to fail
%simultaneously.


Reliability techniques will of course make use of the overlays
for resource localization and monitoring. 
Replicated VMs should be hosted on nodes that have a low
probability to fail simultaneously. Following the previously defined
overlay structure, this can be done through a navigation scheme where
at least one bridge is encountered. Monitoring a replica can then be
done by having a \emph{watcher} in the same local group as the
replica.

%% Another key use of this low-level overlay is proactive replication of VMs,
%% keeping in mind that two identical VMs should be placed in relatively distant
%% nodes, for fault-tolerance reasons (close nodes have a high probability to fail
%% together). Following the defined overlay structure, this can be done through a
%% navigating scheme where at least one bridge is encountered. Monitoring this
%% replica can be done easily by having a \emph{watcher} in the same local group as
%% the replica.




%
%% To this aim, the \discovery system should include, first,  mechanisms in charge of its own reliability. 
%% Such mechanisms are required to avoid losing or corrupting important information
%% regarding the state of the system.  Of course, handling all kinds of failures
%% and implementing a fully resilient operating system is a complex task. Hence,
%% we propose to consider in a first time a crash-stop failure model. In other words, the
%% \discovery system should be able to autonomously restart any service by
%% relocating it on an healthy agent each time it is mandatory.  This implies to
%% define a common design pattern, \textit{i.e.} a set of recommendations, that each
%% service should follow to ensure such characteristics. Besides, in order to
%% ensure that such a pattern may be applied to stateful services, the system can
%% require a Cassandra like system \cite{lakshman:2010} that provides reliable and highly available
%% storage for critical system states. Retrieving the information related to one
%% service need to rely on the kind of mechanisms described in Section~\ref{ssec:p2p}.

%% %
%% In addition to be robust enough, the \discovery system should ensure the
%% reliable execution of virtual environments. The first mechanisms consists in
%% using snapshotting capabilities delivering by virtualization technologies.
%% Concretely, each internal states of a VM should be periodically saved on a
%% persistent storage.  When a crash occurs on a VM, its associated VE can be
%% restarted from the latest consistent states, \textit{i.e.} all VMs of the VE will be
%% resume for their latest snapshot.  As for the other mechanisms, performing VM
%% snapshotting in a large-scale, heterogeneous and widely spread environment is a
%% challenging task. However, we believe that aforementioned mechanisms in charge
%% of the VM images as well as recent proposals \cite{nicolae:2011} might enable
%% to provide such a feature. 
%% Although VM snapshotting provides a first level of reliability, it is not
%% sufficient to ensure high availability of the VE. More advanced mechanisms must
%% be proposed.  Our idea is to include mechanisms based on primary-backup
%% replication techniques. 
%% The basic principle is to have one active replica of the VM (the primary)
%% sending state updates to the other replicas (the backups) periodically. If the
%% primary fails, one of the backup can resume the execution transparently for the
%% outside world. Furthermore since the entire VM is replicated, applications can
%% be run unmodified.  Solutions to replicate VMs inside a cluster have been
%% proposed. However efficiently replicating VMs over a WAN is a huge challenge.
%% Limiting the size of the backup updates\cite{rajagopalan:2012}, and
%% reducing the impact of the required synchronizations on the execution of the
%% primary \cite{gerofi:2012}  are research directions to be further
%% studied. A better understanding of the parts of a VM that really need to be
%% updated is required. It might require to trade transparency for performance by
%% allowing latency-sensitive applications to define which part of their state has
%% to be updated.

%% \ftodo[AL]{Integrate the P2P aspect into the reliability section}
%% \subsubsection*{Monitoring Example}

%% Another key use of this low-level overlay is proactive replication of VMs,
%% keeping in mind that two identical VMs should be placed in relatively distant
%% nodes, for fault-tolerance reasons (close nodes have a high probability to fail
%% together). Following the defined overlay structure, this can be done through a
%% navigating scheme where at least one bridge is encountered. Monitoring this
%% replica can be done easily by having a \emph{watcher} in the same local group as
%% the replica.

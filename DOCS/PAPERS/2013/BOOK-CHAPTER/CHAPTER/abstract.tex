\abstract*{
Although the concept of Micro and Nano Data Centers (DCs) has been proposed to
deliver more efficient as well as sustainable Utility Computing (UC) resources,
the questions of where deploying and how federating thousands of such
facilities are still far to be solved and  the current trend of building larger
and  larger DCs in few strategic locations still prevails.
%
In this chapter, we claim that a new generation of UC platforms can be designed
by juxtaposing the concept of micro/nano DCs with the Internet backbone:
Instead of building and deploying dedicated facilities, we propose to  extend
each network point of presence with additional servers and to operate them
through a unified system in charge of turning a widely distributed
infrastructure into a collection of abstracted computing facilities that
natively matches the geographical dispersal of users. 
%
Delivering such a system is an old objective of the distributed computing community. However, 
unlike previous researches on distributed operating systems, we propose to
consider virtual machines (VMs) instead of processes as the basic element.
System virtualization offers several capabilities that increase the flexibility
of resources management, allowing the investigation of novel autonomic and
decentralized schemes.
}


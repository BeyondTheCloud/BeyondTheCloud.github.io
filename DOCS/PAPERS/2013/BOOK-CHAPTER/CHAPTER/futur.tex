%%%
\section{Future Work/Opportunities\label{sec:future}}

%\subsection{How Evaluating LUC proposals}
%
%We first need to evaluate the quality of our LUC design in terms of performance and fault
%tolerance. For such a large system over a wide infrastructure, getting an analytical model
%is indeed a tough task. To make sure that the overall systems is efficient and reliable,
%simulation and experimentation over an actual platform are the best way to evaluate it. We
%will first rely on simulation to validate independent parts of the overall
%system. Simgrid~\cite{Casanova:2008:SGF:1397760.1398183} will be our platform
%of choice for this evaluation. We are currently adding virtualization
%simulation capabilities to the simulator. Our second evaluation platform will
%be Grid'5000. It provides a testbed supporting experiments on various types of
%distributed systems (high-performance computing, grids, peer-to-peer systems,
%cloud computing, and others), on all layers of the software stack. The core
%testbed currently comprises 10 sites. 
%%Grid'5000 is composed of almost 8000 CPU cores, with various generations of technology, CPUs from one to 12 cores,
%%Myrinet, Infiniband, and 2 GPU clusters). 
%A dedicated 10 Gbps backbone network
%is provided by RENATER. 
%%In order to prevent Grid'5000 machines from being the
%%source of a distributed denial of service, connections from Grid'5000 to the
%%Internet are strictly limited to a list of whitelisted data and software
%%sources, updated on demand. Users are allowed to install their own software
%%stack and run their experiment on a dedicated hardware. Grid'5000 is indeed a
%%Hardware as a Service (HaaS) platform. The communications between clusters can
%%also use virtual networks.
%While the use of RENATER in the Grid'5000
%project was ``only'' to provide users with a dedicated network with several
%functionalities such as XXX, in the DISCOVERY project and in collaboration with RENATER
%engineers, we will deploy a specific testbed on top of their network servers.

\subsection{Geo-diversification as a Key Element}
The Cloud Computing paradigm is changing the way applications are designed.  In
order to benefit from elasticity capabilities of Cloud systems, applications
integrate or leverage mechanisms to provision resources, \ie starting or
stopping VMs, according to their fluctuating needs.
The ConPaaS system~\cite{pierre:2012} is one of the promising systems for elastic Cloud
applications. At the same time, a few projects have started investigating
distributed/collaborative ways of hosting famous applications such as Wikipedia
or Facebook-like  systems by leveraging volunteer computing techniques. 
However, considering that resources provided by end-users were not reliable enough, only few contributions 
have been done yet. 
%
By providing a system that will enable to operate widely spread but more
reliable resources closer to the end-users, the LUC proposal may strongly
benefit to this research area.
Investigating the benefit of locality provisioning (\ie combining elasticity and distributed/collaborative
hosting) is a promising direction for all Web services that are embarrassingly distributed~\cite{church:2008}.
Image sharing systems, such as Google Picasa  or Flickr,  are
examples of applications where leveraging locality will enable to limit network exchanges:
Users could upload their images on a peer that is close to them, and images would be
transferred to other locations only when required (pulling vs. pushing
model).

LUC infrastructures will allow envisioning a wider range of services that may
answer specific SMEs requests such as data archiving or backup solutions, while
significantly reducing the network overhead as well as legal concerns. Moreover, 
it will make the deployment of UC services easier by relieving developers of the burden of dealing with
multi-Cloud vendors.
Of course, this will require software engineering and
middleware advances to easily take advantage of locality. But proposing LUC OS
solutions, such as the  \discovery project, is the mandatory step before
investigating new APIs enabling applications to directly interact with the LUC OS internals. 
%
%A particular issue is to design adequate abstractions that LUC OS has to provide with
%respect to application description.  Indeed, specific interactions have to be set up so
%that applications can leverage LUC OS capabilities, for interactive and batch
%applications. It will require to understand what can/has to be handle at the LUC OS and
%what is the responsibility of the application (or its runtime). The risk is to have
%independent adaption loops at several levels of the software stack. Although a platform such
%as DISCOVERY is designed to provide IaaS offers, we should start to investigate how PaaS
%and SaaS solutions can benefit from the LUC properties directly in their internals by
%delivering the right API enabling applications to directly interact with the DISCOVERY OS.
%
\subsection{Energy, a Primary Concern for Modern Societies}

The energy footprint of current UC infrastructures, and more generally of the
Internet, is a major concern for the society.  By its design and the way
to operate it, a LUC infrastructure will have a smaller impact.
 Moreover, the LUC proposal is an interesting way to
deploy the data furnaces proposal~\cite{liu:hotcloud11}.  Concretely, following
the smart city recommendations (\ie delivering efficient and
sustainable ICT services), the construction of new districts in metropolises
may take advantage of each LUC/Network PoP in order to heat buildings while
operating UC resources remotely by means of a LUC OS. Finally, this idea might
be extended by taking into account recent results about passive data centers,
such as solar-powered
micro-data centers\footnote{\href{http://parasol.cs.rutgers.edu}{\url{http://parasol.cs.rutgers.edu}}}.
The idea behind passive computing facilities is to limit as much as possible
the energy footprint of major hubs and DSLAMS by taking advantage of renewable
energies to power them and by using the heat they product as a source of
energy. Combining such ideas with the LUC approach would allow reaching an
unprecedented level of energy efficiency for UC platforms.


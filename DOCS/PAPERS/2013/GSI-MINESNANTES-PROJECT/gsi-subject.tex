\documentclass[a4paper,11pt]{article}

\usepackage[utf8]{inputenc}
\usepackage[T1]{fontenc}
\usepackage[french]{babel}
% paquet / note EMN
\usepackage{sujetProjet}

\usepackage{xspace}

\usepackage{pdfpages}
\newcommand{\api}{\texttt{API}\xspace}
\newcommand{\vivaldi}{\texttt{Vivaldi}\xspace}
\newcommand{\emn}{EMN}
\newcommand{\eclipse}{\texttt{Eclipse}}
\newcommand{\intellij}{\texttt{Intellij}}
\newcommand{\gsi}{GSI}
\newcommand{\java}{\texttt{Java}\xspace}
\newcommand{\scala}{\texttt{Scala}\xspace}
\newcommand{\uv}{UV}
\newcommand{\xml}{\texttt{XML}}
\newcommand{\discovery}{\texttt{DisCoVEry}\xspace}
\newcommand{\fge}{\fg\ }

\initTitre{UV \og Projet \fge -- Mise-en-oeuvre d'un reseau logique intégrant la notion de distance\\(initiative open-source \discovery)}

\initAbrevTitre{{\discovery~de~\vivaldi}}

%\initSousTitre{}

\initAuteurs{}

%\initDate{22}{8}{2006}

\initSousEntite{Option GSI}
\initCodeSousEntite{GSI}

%\initDestinataires{}

%\initDiffusion{}

%\initCodeNote{UV/}

%\initVersionNote{1.0}

%\initHistorique{}

\pasIndentation

\begin{document}

\begin{couverture}
  \afficherDate
  \afficherTitre
\begin{sujet}

\afficherTuteurs{%
\href{mailto:marin.bertier@irisa.fr}{Marin Bertier} -- 
\href{mailto:cedric.tedeschi@inria.fr}{Cédric Tedeschi} --
\href{mailto:adrien.lebre@inria.fr}{Adrien Lebre} --
\href{mailto:jonathan.pastor@mines}{Jonathan Pastor}
}


\afficherProjet{%
 \discovery~de~\vivaldi -- (IRISA / Mines de Nantes)
 \begin{quote}
 % PRESENTATION DE DISCOVERY ADRIEN
  \end{quote}
}

\afficherSiteWeb{http://beyondtheclouds.github.io}

\afficherIntitule{%
Mise-en-oeuvre d'un reseau logique reposant sur l'agorithme \vivaldi
}

\end{sujet}

\end{couverture}


%\includepdf[pages=-, offset=2cm 0cm]{KeyCopyKey-projet.pdf}

\begin{note}


\paragraph*{Domaines}

\begin{itemize}
 \item Intégration logicielle
 \item Informatique Utilitaire (Cloud Computing)
 \item Langages à objets
 \item Modèle à base d'acteurs
 \item Algorithmique distribuée (réseaux logiques)
 \end{itemize}

\paragraph*{Compétences requises}

\begin{itemize}
 \item Pratique de la programmation (\java/\scala)
 \item Maîtrise de principes de programmation objets
 \item Outil de développement : \eclipse/\intellij
\end{itemize}

\paragraph*{Compétences à acquérir}

\begin{itemize}
 \item Développement d'un réseau logique de type vivaldi
 \item Développement en environement contraint (intégartion des développements dans un contexte logiciel existant)
 \item Développement en mode agile
 \item Contribution à un projet \og~open source~\fg
\end{itemize}

\paragraph*{Contexte~:\\}

 Afin de supporter la demande croissante d'informatique utilitaire (UC) tout en prenant en compte les aspects énergétique et financier, la tendance actuelle consiste à construire des centres de données (ou centrales numériques) de plus en plus grands dans un nombre limité de lieux stratégiques. Cette approche permet sans aucun doute de satisfaire la demande tout en conservant une approche centralisée de la gestion de ces ressources mais elle reste loin de pouvoir fournir des infrastructures de calcul utilitaire efficaces et durables.
   L'initiative de recherche \discovery propose une approche radicalement différente permettant de prendre en compte la localité des demandes dès le départ.
Pour ce faire, elle propose de tirer parti de tous les équipements
disponibles sur l'Internet afin de fournir des infrastructures de calcul
utilitaire  qui permettront de part leur distribution de prendre en compte plus
efficacement la dispersion géographique des utilisateurs et leur demande
toujours croissante [1]. Un des aspects critique pour l'émergence de telles
plates-formes de calcul utilitaire ''local'' (LUC) est la disponibilité de
mécanismes de gestion appropriés exploitant les dernières contributions en terme d'algorithmiques distribués. 
En particulier, il est primordial d'être capable d'estimer la distance entre les noeuds d'une plate-forme à large échelle.

\paragraph*{Le problème~:  distance et localité entre les noeuds\\}

L'algorithme Vivaldi [2] est un algorithme distribué qui permet de donner à un
ensemble de noeuds interconnectés des coordonnées dans un plan en fonction de
leur éloigement physique. Vivaldi s'appuie sur des communications deux à deux,
au cours desquelles les noeuds communiquants estiment leur RTT (round trip
time). A partir de la valeur observée, ils appliquent un calcul simple les
rapprochant ou les éloignant dans le plan. En répétant ce processus sur des
noeuds choisis aléatoirement dans le réseau, les noeuds convergent rapidement
vers leur position finale dans le plan. Cette position peut ensuite être
exploitée pour découvrir des noeuds proches physiquement. En effet, des noeuds
proches dans le plan le seront aussi physiquement.

La notion de distance fournie par Vivaldi permet de construire des réseaux
logiques où la notion de distance physique entre les noeuds est importante. On
souhaite ainsi construire une topologie logique réflétant les distances
physiques entre les noeuds reliés. Il est en particulier nécessaire d'être
capable, depuis n'importe quel noeud du réseau, de retrouver les noeuds les
plus proches physiquement afin de les interconnecter. Une façon de réaliser
cela est de s'appuyer sur une version modifiée de l'algorithme de DIjkstra pour
le calcul du plus court chemin.

\paragraph*{Les objectifs\\}
La problématique du projet est l'implémentation de l'algorithme Vivaldi dans un
contexte logiciel existant, et son expérimentation sur une plate-forme réelle
telle que la plate-forme Grid'5000 (http://www.grid5000.fr).

Le projet sera validé au travers quatres objectifs succesifs:
\begin{itemize}
\item Mise en \oe uvre du réseau logique Vivaldi en suivant un modèle de programmation de type acteur
\item Mise en \oe uvre d'une API permettant d'exploiter cette notion de distance depuis les mécanismes de plus haut niveau. 
\item Mise en \oe uvre d'un mécanisme permettant le parcours du réseau logique de manière efficace (i.e., sur une notion de plus court chemin) en s'appuyant sur l'API bas niveau. 
\item Validation du mécanisme de parcours au sein de la proposition DVMS. 
\end{itemize}

\paragraph*{Les livrables:\\}

\begin{itemize}

\item L0 : Cahier des charges (rappel du contexte et des enjeux, formulation des besoins en développements logiciels)

\item L1 : Rédaction du cahier d'analyse et conception (Définition d'une feuille de route, des solutions logicielles retenues et les différentes étapes qui permettront de valider la bonne avancée du projet). )
\item L2 : implémentation de la solution en tant qu'OverlayActor et validation unitaire.
\item L3 : Implémentation de l'API
\item L4 : Implémentation du mécanisme de parcours
\item L5:  Validation de l'intégration dans le microcosme logicielle \discovery (étude de la pertinence dans le mécanisme DVMS [3]).
\item L6 : synthèse des méthodes et techniques mises en \oe uvre
\end{itemize}

\paragraph*{Méthodologie du projet:\\}
Ce projet sera réalisé suivant une méthode agile. Il devra impliquer une
contribution au projet \og~open source~\fg\ \discovery.


\paragraph*{Bibliographie:\\}
%
\texttt{[1]} Adrien Lebre, Jonathan Pastor, Marin Bertier, Frédéric Desprez, Jonathan Rouzaud-Cornabas, Cédric Tedeschi, Paolo Anedda, Gianluigi Zanetti, Ramon Nou, Toni Cortes, Etienne Rivière, and Thomas Ropars. Beyond The Cloud, How Should Next Generation Utility Computing Infrastructures Be Designed? Research Report RR-8348, Inria, July 2013.\\
\texttt{[2]} Frank Dabek, Russ Cox, M. Frans Kaashoek, Robert Morris: Vivaldi: a decentralized network coordinate system. SIGCOMM 2004\\
\texttt{[3]} Flavien Quesnel, Adrien Lèbre, and Mario Südholt. Cooperative and Reactive Scheduling in Large-Scale Virtualized Platforms with DVMS. Concurrency and Computation: Practice and Experience, December 2012. 

\end{note}
\end{document}



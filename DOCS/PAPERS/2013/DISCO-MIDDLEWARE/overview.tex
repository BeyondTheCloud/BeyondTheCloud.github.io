\section{The LUC Operating System}

To be adopted, the LUC OS should allow end-users to
launch virtual environments (VEs), \textit{i.e.} a set of interconnected VMs,
throughout a  distributed infrastructure as simply as they are used to launch
processes on a local machine, \textit{i.e.}  without the burden of dealing with
resources availability or location. 

Similarly to traditional operating systems (OSes), a LUC OS will be composed of
a significant number of mechanisms.  Investigating all of them in details is
definitely not the purpose of our paper.  In this work, we focus on a rather
large overview of how a new new generation of highly efficient and sustainable
UC  can emerge through advanced system mechanisms.  We describe the main
challenges that should be addressed as well as the main benefits that
may provide such a unified system. 


In particular, we discuss the core of such a LUC
OS, addressing the following objectives:
\begin{description} 
\item Scalability: \discovery must be able to manage hundreds of
  thousands of virtual machines (VMs) running on thousands of
  geographically distributed computing resources, including small and
  medium-sized computing facilities as well as any idle resources that
  their owners would make available. Many of these resources might be
  highly volatile, especially idle resources and resources hosted by
  end-users.
\item Reactivity: To deal with the infrastructure dynamism, \discovery
  should swiftly handle events that require performing particular
  operations, either on virtual or on physical resources, with the
  objective of maximizing the system utilization while meeting QoS expectations of VEs. 
  Reconfiguring  VEs over distributed resources, sometimes spread across wide area networks, or moving VMs, 
  while preserving their active connections, are examples of operations to perform as fast as possible.
\item Resiliency: In addition to the inherent dynamism of the
  infrastructure, failures should be considered as the norm rather
  than the exception at such a scale. The goal is therefore to
  transparently leverage the underlying infrastructure redundancy to
  provide highly available VEs without impacting performance.
\item Sustainability: Although the LUC approach natively reduces the energy
footprint of UC services by minimizing the network impact, it is important to go one
step further by considering energy aspects at each level of the \discovery
system and propose advanced mechanisms in charge of making an optimal usage of each source of energy. 
%Minimizing the energy footprint is a
%  transversal concern that has to be considered at each level of the
%  design of \discovery.
 To achieve such an objective, data related to the energy
  consumption of the VEs  and the computing resources
  as well as the environmental conditions (computer room air conditioning unit, localization of the
  site, etc.) will be taken into account by the system.
\end{description}

Amongst the numerous scientific and technical challenges that should be addressed, 
the lack of a global view of the system will introduce a lot
of complexity. In order to tackle it while addressing the above-mentioned
challenges, we claim that internal mechanisms of the LUC OS should leverage 
the latest contributions in distributed and
cooperative algorithms such as gossip-based approaches and self-* techniques.
These techniques allow fully decentralized, autonomous and self-adapting
control and monitoring mechanisms for complex large-scale systems: Simple
locality-based actions by each of the entities composing the system can lead to
the global emergence of complex and sophisticated behaviors, such as the
self-optimization of resource allocation, or the creation of decentralized
directories. These techniques are starting to be used in well-known large
systems. As an example, the Amazon website leverages gossip techniques in its
Dynamo service \cite{decandia:2007} to create massive scale distributed
indexes and recover from data inconsistencies. . Facebook’s Cassandra massive
scale structured store \cite{lakshman:2010} also leverages gossip-based
techniques for its core operation. \discovery will leverage decentralized and
self-organizing overlays that will reflect the current state of the
infrastructure resources, their characteristics and availabilities, and that
are constructed and operated using gossip-based techniques in order to provide
scalability and flexibility


In addition to the aforementioned objectives, targeting a distributed system
where VM is the elementary granularity requires to deal with important issues
regarding the management of the VM images. Managing VM images in a distributed
way across a WAN is a real challenge that will require to adapt
state-of-the-art techniques such as replication and deduplication. Also,
several mechanisms of \discovery will have to take into account VM images
location, for instance to allocate the right resources to a VE  or to request
VM images prefetching to improve deployment performance or VM migrations.




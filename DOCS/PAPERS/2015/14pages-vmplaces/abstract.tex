% Most current infrastructures for cloud computing leverage static and
% greedy policies for the placement of virtual machines. Such policies
% impede the optimal allocation of resources and the satisfaction of
% operational guarantees like service-level agreements. In recent years,
% more dynamic and often more efficient policies based, \eg on
% consolidation and load balancing techniques, have been developed. Due
% to the underlying complexity of cloud infrastructures, these policies
% are evaluated either using limited scale testbeds/\emph{in-vivo}
% experiments or ad-hoc simulator techniques. These validation
% methodologies are unsatisfactory for two important reasons: they (i)
% do not model precisely enough real productions platforms (size,
% workload representativeness, failure, etc.) and (ii) do not enable the
% fair comparison of different approaches.

Most current infrastructures for cloud computing leverage static and
greedy policies for the placement of virtual machines. Such policies
impede the optimal allocation of resources and the satisfaction of
operational guarantees like service-level agreements. In recent years,
more dynamic and often more efficient policies based, e.g., on
consolidation and load balancing techniques, have been developed. Due
to the underlying complexity of cloud infrastructures, these policies
are evaluated either using limited scale testbeds/in-vivo experiments
or ad-hoc simulator techniques. These validation methodologies are
unsatisfactory for two important reasons: they (i) do not model
precisely enough real productions platforms (size, workload
representativeness, failure, etc.) and (ii) do not enable the fair
comparison of different approaches. More generally, new placement
algorithms are thus continuously proposed without really identifying
the significant benefits of each of them.

In this article, we show how VMPlaceS, a dedicated simulation
framework enables researchers (i) to study and compare VM placement
algorithms , (ii) to detect possible limitations at large scale and
(iii) easily investigate different design choices. Built on top of the
SimGrid simulation platform, VMPlaceS provides programming support to
ease the implementation of placement algorithms and runtime support
dedicated to load injection and execution trace analysis. We
investigate three well known strategies in terms of
reactivity and fault tolerant properties. Diving into the details, we
also identify several modifications that can significantly increase
their performance. We believe that VMPlaceS will allow researchers to
validate the significant benefits of new placement algorithms, thus
accelerating VM placement research results and favouring the transfer
to IaaS production platforms.

% In this article, we propose \vmps, a dedicated simulation framework to
% perform in-depth investigations of VM placement algorithms and compare
% them in a fair way. Built on top of the \sg simulation platform, our
% framework provides programming support to easy the implementation of
% placement algorithms and runtime support dedicated to load injection
% and execution trace analysis. \vmps supports a large set of parameters
% enabling researchers to design simulations representative of
% real-world scenarios. We also report on a validation of our framework
% by evaluating three classes of placement algorithms: centralized,
% hierarchical and fully-distributed ones. We show that \vmps enables
% researchers (i) to study and compare such strategies, (ii) to detect
% possible limitations at large scale and (iii) easily investigate
% different design choices.

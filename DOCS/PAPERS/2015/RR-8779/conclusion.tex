\section{Conclusion and Future Work\label{sec:con}}
\label{sec:conclusion}
Distributing the way Cloud are managed is one solution to favor the
adoption of the distributed cloud model.
%
In this document, we have presented our view of how such distribution can
be achieved. We highlighted that it has however a design cost and it
should be developed over mature and efficient solutions. With this
objective in mind, we chose to design our system, the LUC Operating
System, over OpenStack. This choice presents two advantages: minimizing the development
efforts and maximizing the chance
of being reused by a large community.
As as first step, we modified the Nova SQL backend by a
distributed key/value system.
% and validated this prototype through a
%first series of experiments on top of Grid'5000.
Although a more advanced validation of this
change is required and the question of which metrics to use remains,
this first prototype paves the way toward the distribution of
additional OpenStack services.

%As introduced in Section~\ref{leveraging-openstack}, the other services composing
%OpenStack are historically leveraging a relational database to store their
%inner-states.
% As done with Nova, this relational database may be replaced by a key/value
%data-store.

Among the remaining services, the next candidate
is the image service Glance. Indeed, as its images are already stored in fully distributed cloud storage
software (SWIFT or CEPH), the
next step to reach a fully distributed functionning with Glance is to apply the same
strategy that we did with Nova. On the other hand, the situation may be different with
some other services: Neutron works with drivers that may not be intented to work in a
distributed way.  In such situation alternatives will have to be
found.
%Instead of redeploying a complete installation, they will be able to leverage IT resources and specific devices
%such as computer room air conditioning units, inverters or redundant power supply.
%
%% DIRE QUE LE FAIT DE POUSSER VERS OPENSTACK EST SUPER BIEN CAR CA VA
%% PERMETTRE D'AVOIR UN IMPACT SI ON GAGNE
% Finally, to maximize the chance of being reused by a large community,
% a LUC OS
% should enable an easy integration with one or several existing IaaS cloud
% managers. In our case we have chosen to leverage the OpenStack project: as a
% result mechanisms of the LUC-OS will integrate well with existing clouds that
% are based on OpenStack.

% As this project is successful and is used by a large community, we propose that
% instead of reinventing the wheel by developing each components of the LUC-OS
% from scratch, to leverage the OpenStack project. This strategy would enable to
% focus the effort on the key issues such as the distributed functionning, the
% fault tolerance mechanisms and the organization of efficient collaborations
% between nodes of the infrastructure.

%\subsection{Deeper validation with larger testbed}

% Several other researches and developments needs to be performed to get the distributed
% environment we have in mind.

% First, even if this article has demonstrated that the relational database used by Nova
% could be replaced by a non relational key/value store,

%  a more advanced validation of this
% change is required and the question of which metrics to use remains. On one hand, a larger
% deployment involving more geographical sites would be more demonstrative, on the other
% hand the changes we introduced have been have been motivated by more than overall
% scalabity, but also fault tolerance and reduction of response times when processing API
% requests.
%
%\subsection{Distributing the remaining services of OpenStack}
%
% As introduced in Section~\ref{leveraging-openstack}, the other services composing
% OpenStack are historically leveraging a relational database to store their
% inner-states. As done with Nova, this relational database may be replaced by a key/value
% data-store. Among the remaining services, the next candidate is the image service Glance:
% as its images are already stored in fully distributed cloud storage software (SWIFT), the
% next step to reach a fully distributed functionning with Glance is to apply the same
% strategy that we did with Nova. On the other hand, the situation may be different with
% some other services: Neutron works with drivers that may be intented to work in a
% distributed way.  In such situation alternatives have to be found.

%\subsection{Locality aware objects in OpenStack}

Finally, having a wan-wide infrastructure can be source of networking overheads: some objects
manipulated by OpenStack are subject to be manipulated by any service of the deployed
controllers, and by extension should be visible to any of the controllers. On the other
hand, some objects may benefit from a restrained visibility: if a user has build an
OpenStack project (tenant) that is based on few sites, appart from data-replication there
is no need for storing objects related to this project on external sites. Restraining the
storage of such objects according to visibility rules would enable to save network
bandwidth and to settle policies for applications such as privacy and efficient data-
replication.

We believe, however, that adressing all these challenges are key
elements to promote a new generation of cloud computing more
sustainable and efficient. Indeed, by revising OpenStack in order to
make it natively cooperative, it would enable Internet Service Providers and other
institutions in charge of operating a network backbone to build an
extreme-scale LUC infrastructure with a limited additional cost. The
interest of important actors such as Orange Labs that has officially
announced its support to the initiative is an excellent sign of the
importance of our action.

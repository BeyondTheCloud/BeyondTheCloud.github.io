\section{Related Work}
\label{sec:related}

%\AL[AL]{Add few lines related to Google scheduling simulator - https://code.google.com/p/cluster-scheduler-simulator/}
%\AL[AL]{Add few lines about FlexCloud - http://arxiv.org/pdf/1501.05789v1.pdf}
Simulator toolkits that have been proposed to address CC
concerns~\cite{cloudsim,CC13,DGSIM,greencloud,icancloud} can be
classified into two categories.  The first corresponds to ad-hoc
simulators that have been developed to address one particular
concern. For instance, CReST~\cite{CC13} is a discrete event
simulation toolkit built for Cloud provisioning algorithms. If ad-hoc
simulators allow some characteristics of the behaviors of the system
to be analyzed, they do not consider the implication of the different
layers, which can lead to non-representative results. Moreover, most
ad-hoc solutions are developed for one shot analyses. That is, there
is no effort to release them as a complete and reusable tool for the
scientific community. The second category
\cite{cloudsim,greencloud,icancloud} corresponds to more generic cloud
simulator toolkits (\ie they have been designed to address multiple CC
challenges). However, they focus mainly on the API and not on the
model of the different mechanisms of CC systems. For instance,
CloudSim~\cite{cloudsim}, which has been widely used to validate
algorithms and applications in different scientific publications, is
based on a top-down viewpoint of cloud environments.  That is, there
are no articles that properly validate the different models it relies
on: a migration time is simply (and often imprecisely) calculated by
dividing VM memory sizes by network bandwidth values.
%Such a model cannot correctly simulate many real
%environments where workloads perform substantial memory writes.
In addition to be subject to inaccuracies at the low level, available
cloud simulator toolkits often use oversimplified models for
virtualization technologies, also leading to non-representative
results. As highlighted throughout this article, we have chosen to
build \vmps on top of \sg in order to build a generic tool that
benefits from the accuracy of its models related to virtualization
abstractions~\cite{Hirofuchi:2013:ALM:2568486.2568524}.



%%% Local Variables:
%%% mode: latex
%%% TeX-master: "main"
%%% End:

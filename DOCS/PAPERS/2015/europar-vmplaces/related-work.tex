\section{Related Work}
\label{sec:related}
%\AL[AL]{Add few lines related to Google scheduling simulator - https://code.google.com/p/cluster-scheduler-simulator/}
%\AL[AL]{Add few lines about FlexCloud - http://arxiv.org/pdf/1501.05789v1.pdf}
Several simulator toolkits have been proposed since the last years in
order to adress CC concerns~\cite{CC13,DGSIM,cloudsim,icancloud,greencloud}.  They can be classified into two categories: The first
one corresponds to ad-hoc simulators that have been developped to
address a particular concern. For instance, CReST~\cite{CC13} is a
discrete event simulation toolkit built to evaluate Cloud provisioning
algorithms. If ad-hoc simulators enable to provide some trends
regarding the bevahiours of the system, they do consider the
implication of the different layers, which can lead to non
representative results at the end. Moreover, such ad-hoc solutions are
developped for one shot and thus, they are not available for the
scientific community. The second category \cite{icancloud,greencloud,
  cloudsim} corresponds to more generic cloud simulator toolkits (\ie
they have been designed to adress a majority of CC
challenges). However, they have focused mainly on the API and not on
the model of the different mechanisms of CC systems.

For instance, CloudSim~\cite{cloudsim}, which has been widely used to
validate algorithms and applications in different scientific
publications, is based on a relatively top-down viewpoint of cloud
environments.  That is, there is no papers that properly validate the
different models it relies on: a migration time is calculated by
dividing a VM memory size by a network bandwidth.
%Such a model cannot correctly simulate many real
%environments where workloads perform substantial memory writes.
 In addition to having inaccuracy weaknesses at the low level, available cloud
simulator toolkits over simplified the model for the virtualization
technologies, leading also to non representation results at the
end. As highlighted several times throughout this document, we chose to
build \vmps on top of \sg in order to benefit fromt its accuracy of
its models related to virtualization abstractions~\cite{Hirofuchi:2013:ALM:2568486.2568524}.
